%
% Copyright (c) 1996, 2004, 2006, 2009, 2014
% Tama Communications Corporation. All rights reserved.
%
% Redistribution and use in source and binary forms, with or without
% modification, are permitted provided that the following conditions
% are met:
% 1. Redistributions of source code must retain the above copyright
%    notice, this list of conditions and the following disclaimer.
% 2. Redistributions in binary form must reproduce the above copyright
%    notice, this list of conditions and the following disclaimer in the
%    documentation and/or other materials provided with the distribution.
%
% THIS SOFTWARE IS PROVIDED BY THE AUTHOR AND CONTRIBUTORS ``AS IS'' AND
% ANY EXPRESS OR IMPLIED WARRANTIES, INCLUDING, BUT NOT LIMITED TO, THE
% IMPLIED WARRANTIES OF MERCHANTABILITY AND FITNESS FOR A PARTICULAR PURPOSE
% ARE DISCLAIMED.  IN NO EVENT SHALL THE AUTHOR OR CONTRIBUTORS BE LIABLE
% FOR ANY DIRECT, INDIRECT, INCIDENTAL, SPECIAL, EXEMPLARY, OR CONSEQUENTIAL
% DAMAGES (INCLUDING, BUT NOT LIMITED TO, PROCUREMENT OF SUBSTITUTE GOODS
% OR SERVICES; LOSS OF USE, DATA, OR PROFITS; OR BUSINESS INTERRUPTION)
% HOWEVER CAUSED AND ON ANY THEORY OF LIABILITY, WHETHER IN CONTRACT, STRICT
% LIABILITY, OR TORT (INCLUDING NEGLIGENCE OR OTHERWISE) ARISING IN ANY WAY
% OUT OF THE USE OF THIS SOFTWARE, EVEN IF ADVISED OF THE POSSIBILITY OF
% SUCH DAMAGE.
%
\documentclass[a4]{jarticle}
\usepackage{rireki}
%
% オプション
%
% 下記のオプションが利用可能です。
%
% \空行挿入		% 学歴と職歴の間に空行を挿入します
\toA4			% A4用紙にB5 のトンボ付きで出力します
%			(履歴書のフォーマットはB5サイズ固定です)
\begin{document}
%
% ID情報
%
\姓{佐々木}
\名{裕}
\姓読み{ささき}
\名読み{ひろし}
\性別{男}					% 男|女
\誕生日{1959 年 3 月 17 日}
\年齢{62}
%
% 顔写真
%
% 画像ファイルにはEPS フォーマット・縦横比4:3 のものをご使用ください。
% 縦を4cm に調整し、縦横比を変更せずに印刷します。
% 次のように指定します。
% \顔写真{photo.eps}
%
\顔写真{photo.eps}
%
% 現住所
%
\現住所郵便番号{468-0075}
\現住所{愛知県名古屋市天白区御幸山719-2}
\現住所読み{あいちけんなごやしてんぱくくみゆきやま}
\現住所市外局番{052}
\現住所電話番号{836-3487}
\現住所呼び出し{}
%
% 連絡先
%
\連絡先郵便番号{}
\連絡先{\tt e-mail: hsasaki@xj.commufa.jp}
\連絡先読み{}
\連絡先市外局番{}
\連絡先電話番号{}
\連絡先呼び出し{}
%
% 学歴、職歴
%
% 学歴、職歴を年月順に列挙してください。合計20個まで記入出来ます。
% 20個を超える部分は印刷されませんので、ご注意ください。
% 印刷順は、学歴=>職歴の順になります。
%
\学歴{1974}{4}{大分県立別府鶴見が丘高等学校 入学}      % {年}{月}{内容}
\学歴{1977}{3}{大分県立別府鶴見が丘高等学校 卒業}
\学歴{1977}{4}{北海道大学 入学}
\学歴{1984}{3}{北海道大学 工学部合成化学工学科 卒業}
\学歴{1984}{4}{北海道大学大学院工学研究科修士課程 合成化学工学専攻 入学}
\学歴{1986}{3}{北海道大学大学院工学研究科修士課程 合成化学工学専攻 修了}
\職歴{1986}{4}{東亞合成株式会社 入社}
\学歴{1992}{1}{米国 Rennselaer Polytechnic Institute(海外留学)}
\学歴{1993}{3}{米国 Rennselaer Polytechnic Institute より帰国}
\学歴{1994}{4}{北海道大学地球環境科学研究科博士後期課程 入学}
\学歴{1996}{3}{北海道大学地球環境科学研究科博士後期課程 修了}
\職歴{2021}{10}{現在にいたる}
%
% 資格
%
% 資格を取得年月順に列挙してください。9つまで記入できます。
% 9つを超える部分は印刷されませんので、ご注意ください。
%
\資格{1981}{7}{普通自動車一種免許}            % {取得年}{取得月}{資格}
\資格{1988}{3}{甲種危険物取扱者}            % {取得年}{取得月}{資格}
\資格{1996}{12}{TOEIC 835 点}
%
% 個人情報
%
% 志望の動機と本人希望記入欄はlatex のコマンドを記述できます。
%
\志望の動機{
	\begin{tabular}{ll}
	{\gt 趣味・特技} & テニス、読書\\
	{\gt 高分子関連} & 高分子材料を中心とした機能性材料の設計・合成・評価\\
	{\gt 理論的アプローチ} & 統計力学、高分子物理を中心とした理論的アプローチに関する広範な知見\\
	{\gt 研究・開発全般} & 研究・開発に関わるコンセプトの集約化とそのプレゼンテーション\\
	{\gt シミュレーション} & SCF 計算、MD 計算、DPD 等の各種シミュレーションに関する知見\\
	{\gt 分析・評価技術} & 分析、観察、粘弾性測定等の各種評価技術\\
	\end{tabular}
}
\本人希望記入欄{

職務経歴書に示しましたような幅広い経験を生かして、あらゆる世代の研究員をも巻き込んだ形で各種の研究・開発業務を推進していくことができる自信があります。

%	私が希望する仕事の条件は下記の通りです。
%	\begin{itemize}
%	\item ◯◯◯◯◯◯◯◯◯◯◯◯◯◯◯◯◯
%	\item ◯◯◯◯◯◯◯◯◯◯◯◯◯◯◯◯◯
%	\item ◯◯◯◯◯◯◯◯◯◯◯◯◯◯◯◯◯
%	\end{itemize}
}
%
% その他
%
\通勤時間{}
\扶養家族数{2}					% 人数(配偶者を除きます)
\配偶者{あり}					% あり|なし
\配偶者の扶養義務{あり}	

\サイン{Your Signature}

\end{document}
