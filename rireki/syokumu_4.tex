%
% LaTeXのクラスの指定や使用するパッケージの指定
% 通常は変更する必要はない
%
\documentclass[a4j]{jsarticle}
\usepackage{shokureki}

%%%%%%%%%%%%%%%%%%%%%%%%%%%%%%%%%%%%%%%%%%%%%%%%%%%%%
%----------------------------------------------------
% 表示設定
% 以下の表示設定をコメントアウトしたり有効にしたりすることで
% 職務経歴書の表示形式を変更できる。
%----------------------------------------------------
%\compact{0.9}		% コンパクト出力 (スペースを節約して出力)
			% 数値は行間指定 数値が大きいほど行間が広がる。標準は1.0
\reverseDate{yes}	% \fromto マクロの上下指定。yesだと矢印が上向きになる
			% 最近の職務を上に書く場合に有益
\pagestyle{plain}	% ページ番号を表示する
%\listProject{yes}	% プロジェクトリストに表を使わない
%\breakBox{yes}		% ページをまたぐ枠を使用する(表を使わない指定の時に意味がある)

%%%%%%%%%%%%%%%%%%%%%%%%%%%%%%%%%%%%%%%%%%%%%%%%%%%%%
%----------------------------------------------------
% タイトルの指定
% \rubyはふりがなルビの指定
% \dateの\西暦を取り除くと元号表示になる
% また特定の日時を指定することで現在以外の日付を表示できる。
%----------------------------------------------------
\title{職務経歴書}
\author{\ruby{佐々木}{ささき} \ruby{裕}{ひろし}}
\date{
%\西暦
\today 現在}
%----------------------------------------------------
%ドキュメント内で使用できるマクロについてはこのファイルの末尾で解説




%%%%%%%%%%%%%%%%%%%%%%%%%%%%%%%%%%%%%%%%%%%%%%%%%%%%%
%====================================================
%
% ここからドキュメントスタート
%
%====================================================
\begin{document}
%----------------------------------------------------
%  タイトルの出力
%----------------------------------------------------
\maketitle% 通常タイトル
%\makeshorttitle% 短縮タイトル

\vspace{-5mm}


%%%%%%%%%%%%%%%%%%%%%%%%%%%%%%%%%%%%%%%%%%%%%%%%%%%%%
%----------------------------------------------------
%  最初にあると良いとされる経歴の要約部分
%  読み手に準備をさせる効果があるという。
%----------------------------------------------------
\要約{%

1986 年に東亞合成株式会社に入社後、光硬化型材料を手始めとした各種の新規機能製材用の研究・開発に従事し、担当テーマの完遂および新規材料の創成を通して特許出願による権利の取得を行ってきました。
また、以下の主たる研究開発に関しては学会などでも多数の発表を行い高い評価を得ております
\footnote
{
査読付き論文 11 報、国際会議論文 15 報、著書(分筆) 11 冊、和文解説 3 報。
国内特許 29、海外特許 4、国内学会での発表多数。
}。
\begin{description}
\item[シャドウマスクマスキング剤]
当時のテレビに用いられていたブラウン管に必要欠くべからざる部品であったシャドウマスクはエッチング工程を経て製造されており、その製造時にはマスキング剤が必須であった。
その生産性を大幅に向上しうる光硬化型マスキング剤の開発を開発グループの中心となって担当し、その根幹となる技術開発を完遂できました
\footnote
{
この開発の過程において、特許 2649622 号「裏止め材用硬化性樹脂組成物及びシヤドウマスクの製造方法」、特許 2649621 号「同前」、特許 2592733 号「硬化性樹脂組成物」などの特許を取得しています。

また、この開発の成果として、後に、東亞合成株式会社と大日本スクリーン株式会社とのシャドウマスク製造に係わる合弁会社の設立に至っています。
}。

\item[オキセタン化合物]
引き続いての米国留学時に、新規な重合開始方法であるカチオン重合型紫外線硬化型材料についての研究を開始し、オキセタン化合物の新規材料としての有用性を見出しました。
帰国後には、オキセタン樹脂の重合性および硬化物物性に関する基礎的研究から、その工業的な応用に至るまでの研究開発を主として担当し、年商数億程度の規模の事業化まで育てることに成功いたしました
\footnote
{
これらに関連して、特許 2679586 号「活性エネルギ-線硬化型組成物」、特許 3161583 号「活性エネルギ-線硬化型組成物」、などのオキセタン化合物の工業的利用における基本的な特許を取得しています。

なお、1996 年にはオキセタン関連の研究により、北海道大学大学院地球環境科学研究科において博士の学位を取得しています。
}
\footnote
{
なお、上記二件の研究・開発の結果にたいして、社長表彰をそれぞれ受けています。
}。

\item[ホログラムメモリー]
光硬化型材料の開発により得た知見の応用技術として、当時注目を集めていたホログラムメモリーへの応用を目指した研究・開発にも参画いたしました。

\item[ネットワークポリマーのゴム弾性]
光硬化型材料に限らない各種の高分子材料の物性設計を通して体系化できてきた知見をまとめ、材料の耐久性向上にも強く関連するゴム弾性挙動の解析についての研究も進めております
%\footnote{
%これらの結果については、高分子学会、レオロジー学会等での発表も行っております。
%}
。


\end{description}

上記の各種研究開発に従事した後に、基盤技術研究所において、それまでの研究・開発業務で培ってきました各種の評価技術について自身の知見の深化、および、後進の指導に努めてまいりました。
さらに、粘弾性測定やシミュレーション等の新たな評価技術についての知見も深めてきております。

これまでの研究、開発、および、所長としての経験により身に付けてきた技術及び考え方は、他の分野の研究・開発においても十分に応用できるものと信じております。

}

%%%%%%%%%%%%%%%%%%%%%%%%%%%%%%%%%%%%%%%%%%%%%%%%%%%%
%----------------------------------------------------
% 在籍してきた企業の情報を掲載
% どのような職場で働いてきたか、相手に想像させる材料を提供
%
% 職歴リスト環境
%       形式:
%       \begin{職歴リスト}{タイトル}
%       {項目1のタイトル}{項目2のタイトル}{項目3のタイトル}
%
% 職歴リスト環境では職歴マクロが使用できる
%       形式:
%       \職歴{期間}{勤務機関(会社)の情報}{所属・部門・担当情報}
%       項目内で\\による改行が可能
%----------------------------------------------------
% \begin{職歴リスト}{職務経歴概略}{期間}{勤務会社}{所属/部門/役職}
% \職歴{\fromto{1986/4}{2020/9}}
% {
% \begin{箇条書き}
% \項目{東亞合成株式会社}
% \項目{業務内容:化学系製造業}
% \項目{資本金:208 億}
% %\項目{年商:単体 720億 連結1,500億}
% %\項目{従業員数:単独 1197 連結 2,429}
% \end{箇条書き}
% }
% {%所属等
% \begin{説明リスト}
% \説明{所属}{正社員}
% \説明{部門}{研究企画開発部}
% \説明{担当}{チームリーダー}
% \end{説明リスト}
% }

% %
% \職歴{\fromto{1993/4}{1996/3}}
% {%会社情報
% \begin{箇条書き}
% \項目{北海道大学大学院 地球環境科学専攻 博士課程後期}
% \項目{指導教官:覚知教授}
% \end{箇条書き}
% }
% {%所属等
% \begin{説明リスト}
% \説明{所属}{在職で博士課程に進学}
% \説明{部門}{オキセタン関連の研究に従事}
% %\説明{担当}{開発部員}
% \end{説明リスト}
% }
% %
% % 職歴リスト環境の終了指定 
% %	職歴リストを終了させるのに必須の指定。
% %
% \end{職歴リスト}

\newpage

%%%%%%%%%%%%%%%%%%%%%%%%%%%%%%%%%%%%%%%%%%%%%%%%%%%%%
%----------------------------------------------------
% 職務経歴書の中心的な部分であるプロジェクトリスト
% 各プロジェクトごとに4つの項目が書けるものとする。
% その項目はそれぞれ以下のようなものであることを想定している。
% 項目1:期間
% 項目2:プロジェクト内容
% 項目3:プロジェクト環境
% 項目4:プロジェクト内での役割、プロジェクト規模
%
% プロジェクトリスト環境
%       形式:
%       \begin{プロジェクトリスト}{タイトル}
%       {項目1のタイトル}{項目2のタイトル}{項目3のタイトル}{項目4のタイトル}
%       プロジェクトリスト環境は \end{プロジェクトリスト}で終了する。
%
% プロジェクトリスト環境内部では以下のマクロが使用できる
%       \所属 マクロ
%       形式:
%       \所属{所属会社情報}
% プロジェクトリストの中で後に続くプロジェクトがどの所属会社で
% なされたかを示すために挿入される会社名を含む行を書き出す
%
%       \プロジェクト マクロ
%       形式:
%       \プロジェクト{期間}{プロジェクト内容}{環境OS情報}{役割・規模等情報}
% プロジェクトリストに挿入されるプロジェクトの実体
%       項目内で\\による改行が可能
%
%----------------------------------------------------
\begin{プロジェクトリスト}{業務経験}
        {期間}{業務内容}{関連技術}{役割/規模/担当}
%
\所属{東亞合成株式会社}
\プロジェクト{\fromto{2016/1}{2022/1}}% プロジェクト期間
{% プロジェクト内容
R\&D センター(名古屋)に復帰し、センター長直属として、以下の業務に従事

\begin{箇条書き}
\項目{研究;複数の大学と連携し、新規接着剤に向けた革新的ネットワークポリマーの研究を推進}
%\項目{企画:大学、各種学会からの情報収集に基づき、JACIでの国家プロジェクトの発案に参画}
%\項目{教育:日化協において、化学系博士向けの教育についての企画立案に参画}
\end{箇条書き}
}
{% 環境情報
\begin{説明リスト}
\説明{分析}{特になし}
\end{説明リスト}
}
{% 規模・役割等情報
\begin{箇条書き}
\項目{社外の各種機関との連携の推進}
\項目{革新的ネットワークポリマーの研究}
%\項目{教育:日化協において、化学系博士向けの教育についての企画立案に参画}
\end{箇条書き}

%\begin{説明リスト}
%\説明{役割}{社外の各種機関との連携の推進}
%\説明{役割}{新規テーマの探索および提案}
%\end{説明リスト}
}
%
\プロジェクト{\fromto{2015/4}{2015/12}}% プロジェクト期間
{% プロジェクト内容
東京に拠点を移し、以下の業務に従事

\begin{箇条書き}
\項目{研究;複数の大学と連携し、新規接着剤に向けた革新的ネットワークポリマーの研究を推進}
\項目{企画:大学、各種学会からの情報収集に基づき、JACIでの国家プロジェクトの発案に参画}
\項目{教育:日化協において、化学系博士向けの教育についての企画立案に参画}
\end{箇条書き}
}
{% 環境情報
\begin{説明リスト}
\説明{分析}{特になし}
\end{説明リスト}
}
{% 規模・役割等情報
\begin{箇条書き}
\項目{社外の各種機関との連携の推進}
\項目{新規テーマの探索および提案}
%\項目{教育:日化協において、化学系博士向けの教育についての企画立案に参画}
\end{箇条書き}

%\begin{説明リスト}
%\説明{役割}{社外の各種機関との連携の推進}
%\説明{役割}{新規テーマの探索および提案}
%\end{説明リスト}
}
%
\プロジェクト{\fromto{2013/9}{2015/3}}% プロジェクト期間
{% プロジェクト内容
R\&D センター(名古屋)において、研究開発本部長直属として、以下の業務に従事

\begin{箇条書き}
\項目{研究;新規複合化材料および新規ネットワークポリマーの探索}
\項目{教育:社内において、統計基礎、各種物理関連事項の教育}
\項目{企画:JACI にてプロジェクト部会員として国家プロジェクトの原案の提案}
\end{箇条書き}
}
{% 環境情報
\begin{説明リスト}
\説明{分析}{特になし}
\end{説明リスト}
}
{% 規模・役割等情報
\begin{説明リスト}
\説明{役割}{チームリーダー}
\説明{規模}{チーム員:5 名}
\end{説明リスト}
}
%
% プロジェクトリスト環境ではいくつでもプロジェクトを挿入できる
%
\プロジェクト{\fromto{2009/4}{2013/8}}
{% 
基盤技術研究所所長として、基盤的研究全般に関わる研究活動の推進に従事

\begin{箇条書き}
\項目{基盤技術研究所が所管する「分析、観察、物性評価、探索研究等」に関わるマネージメント}
\項目{R\&D センター全体の研究・開発活動の推進}

\end{箇条書き}
}
{
\begin{説明リスト}
\説明{分析}{各種分析技術}
\説明{観察}{各種観察技術}
\説明{評価}{各種評価技術}
\end{説明リスト}
}
{
\begin{説明リスト}
\説明{役割}{所長}
\説明{規模}{所員:約 40 名}
%\説明{担当}{マネージメント}
\end{説明リスト}
}
%
\プロジェクト{\fromto{2005/4}{2009/3}}
{% 
分析研究所 ($\Rightarrow$ 2008 年に基盤技術研究所に改組)において、物性評価技術に関するチームリーダーとして研究活動に従事

\begin{箇条書き}
\項目{各種物性評価技術に関する研究活動の推進}
\項目{社内の開発グループの研究・開発活動のサポート}
\end{箇条書き}
}
{
\begin{説明リスト}
\説明{分析}{各種反応性材料の評価}
\説明{観察}{粘弾性測定}
\説明{評価}{シミュレーション技術}
\end{説明リスト}
}
{
\begin{説明リスト}
\説明{役割}{チームリーダー}
\説明{規模}{チーム員:4 $\sim$ 6 名}
%\説明{担当}{マネージメント}
\end{説明リスト}
}
%
%
\プロジェクト{\fromto{2003/4}{2005/3}}
{% 
社内プロジェクト「ホログラムメモリー関連技術」に参画し、光硬化材料のメモリー材料への応用研究に従事

\begin{箇条書き}
\項目{ホログラム関連の光学技術に関する研究活動の推進}
\項目{微細光重合技術に関する研究の推進}
\end{箇条書き}
}
{
\begin{説明リスト}
\説明{評価}{光学シミュレーション}
\説明{分析}{光硬化材料の評価}
\説明{観察}{粘弾性測定}
\end{説明リスト}
}
{
\begin{説明リスト}
\説明{役割}{チームリーダー}
\説明{規模}{チーム員:4 名}
%\説明{担当}{マネージメント}
\end{説明リスト}
}
%
%
\プロジェクト{\fromto{1993/4}{2003/3}}
{% 
社内プロジェクト「オキセタン実用化プロジェクト」に参画し、オキセタン樹脂の実用化に向けた研究開発に従事

\begin{箇条書き}
\項目{新規材料の事業化に必須となる「オキセタン樹脂特性の明確化」を推進}
\項目{学会、論文等の発表による「オキセタン樹脂特性」の公知化を推進}
\end{箇条書き}
}
{
\begin{説明リスト}
\説明{分析}{各種反応性材料の評価}
\説明{観察}{粘弾性測定}
\説明{評価}{分子軌道法 (MO) 計算による反応性予測}
\end{説明リスト}
}
{
\begin{説明リスト}
\説明{役割}{チームリーダー}
\説明{規模}{チーム員:4 $\sim$ 6 名}
%\説明{担当}{マネージメント}
\end{説明リスト}
}
%
%%%%%%%%%%%%%%%%%%%%%%
\所属{米国レンセラー大学 (Rensselaer Polytechnic Institute)}
%
\プロジェクト{\fromto{1991/12}{1993/3}}
{% 
米国レンセラー大学に留学(クリベロ教授)

\begin{箇条書き}
\項目{オキセタン材料に関する合成技術研究}
\項目{硬化特性等の評価}
\end{箇条書き}
}
{%
\begin{説明リスト}
\説明{OS}{各種合成技術}
\説明{DB}{反応性評価}
\説明{言語}{各種分析技術}
\end{説明リスト}
}
{%
\begin{説明リスト}
\説明{役割}{研究者}
\end{説明リスト}
}
%
%%%%%%%%%%%%%%%%%%%%%%
\所属{東亞合成株式会社}
%
\プロジェクト{\fromto{1987/12}{1991/11}}
{% 
ラジカル重合型紫外線硬化型材料の応用技術として、「シャドウマスクマスキング材の開発」に従事

\begin{箇条書き}
\項目{各種物性評価によるマスキング剤の要求特性の明確化}
\項目{実ラインでのプロトタイプの実用化を推進}
\end{箇条書き}
}
{%
\begin{説明リスト}
\説明{OS}{各種合成技術}
\説明{DB}{反応性評価}
\説明{言語}{各種分析技術}
\end{説明リスト}
}
{%
\begin{説明リスト}
\説明{役割}{研究者}
\end{説明リスト}
}
%
\プロジェクト{\fromto{1986/4}{1987/11}}
{% 
「ラジカル重合型紫外線硬化型材料」の開発に従事

\begin{箇条書き}
\項目{各種の研究開発業務および新規材料の探索}
\end{箇条書き}
}
{%
\begin{説明リスト}
\説明{OS}{各種合成技術}
\説明{DB}{反応性評価}
\説明{言語}{各種分析技術}
\end{説明リスト}
}
{%
\begin{説明リスト}
\説明{役割}{研究者}
\end{説明リスト}
}
%	
% プロジェクトリストを終了するには
% 以下の\end{プロジェクトリスト}を挿入することが必要
%
\end{プロジェクトリスト}

%\newpage

%%%%%%%%%%%%%%%%%%%%%%%%%%%%%%%%%%%%%%%%%%%%%%%%%%%%
%----------------------------------------------------
% 使用環境・言語等のまとめ情報
%       形式:
%       \begin{経験リスト}{タイトル}
%
% 経験リスト環境では経験マクロが使用できる
%       形式:
%       \経験{経験項目名}{経験したもの}
%       項目内で\\による改行が可能
%----------------------------------------------------
\begin{経験リスト}{研究・開発に活かせる経験・知識・技術}
\経験{合成技術}{蒸留操作等を含む各種有機合成技術、クロマト操作による分取技術、真空ライン操作を含む各種重合技術}
\経験{分析技術}{NMR, IR 等の各種分光分析技術、液クロおよびガスクロ等の各種クロマト分析技術、各種顕微測定による観察技術}
\経験{評価技術}{反応性評価技術(IR、レオメーター)、粘弾性(レオロジー)測定技術、動的および静的光散乱技術、各種熱分析技術}
\経験{シミュレーション}{分子軌道法 (MO) 計算、分子動力学 (MD) 計算、スリップリンク法を用いた高分子のフロー計算、密度汎関数法による高分子の相分離構造評価、有限要素法による応力ひずみ特性評価}
\経験{物理的な知識}{統計力学、高分子物理を中心とした理論的アプローチに関する広範な知見}
\経験{経験}{紫外線硬化型材料の合成および配合に関する広範な知識}
\経験{その他}{研究・開発に関わるコンセプトの集約化とそのプレゼンテーション}
\end{経験リスト}

%%%%%%%%%%%%%%%%%%%%%%%%%%%%%%%%%%%%%%%%%%%%%%%%%%%
%----------------------------------------------------
% 資格リスト環境
%       形式:
%       \begin{資格リスト}{タイトル}
%       {項目1のタイトル}{項目2のタイトル}
%
% 資格リスト環境では資格マクロが使用できる
%       形式:
%       \資格{取得年月}{資格内容などの説明}
%       項目内で\\による改行が可能
%----------------------------------------------------
\begin{資格リスト}{資格/語学等}{取得年月}{資格内容等}
%
% 資格リストの中では\資格 マクロはいくつでも挿入できる
%
\資格{1988 年 6 月}{甲種危険物取扱者}
\資格{1994 年 8 月}{TOEIC 835 英語でのコミュニケーションに支障なし}
\資格{1996 年 3 月}{北海道大学地球環境科学研究科博士後期課程修了}
%
% 資格リスト環境は以下の\end{資格リスト}で終了させる必要がある
%
\end{資格リスト}


%%%%%%%%%%%%%%%%%%%%%%%%%%%%%%%%%%%%%%%%%%%%%%%%%%%%
%----------------------------------------------------
% 自己PR環境
% 自己PRは、例文にあるように言いたいことに見出しをつけるとわかりやすいとされる
% また根拠となる具体的な事実について述べると説得力が増す。
% 自己PR環境
%       形式:
%       \begin{自己PR}{タイトル}
%
% 自己PR環境では資格マクロが使用できる
%       形式:
%       \PR{サブタイトル}{PRの内容}
%----------------------------------------------------
\begin{自己PR}{自己PR}
\PR{各種反応性材料(特に光硬化型材料)に関する経験}{%
大学時代に専門として学んできた機能性高分子の合成に関する知識をベースとして、新規紫外線硬化型樹脂の探索およびシャドウマスクマスキング剤として使用可能な紫外線硬化型材料の配合について開発を行うことができました。
また、米国留学時には、オキセタン樹脂の新規光カチオン硬化型材料としての新たな可能性を見出すことにも成功しております。

これらの経験を通して、接着剤や塗料に応用可能な各種の反応性材料、とりわけ、光硬化型材料に関する広範な知識を身に付けてきましたので、更なる新規モノマーの合成、および、各種の配合による配合物の設計というような幅広いアプローチについても自身が中心となって推進、さらに、後進の指導にも活かしていくことができると考えております。
}
\PR{新規材料の事業化に関する経験}{%
上述のシャドウマスクマスキング剤は、顧客の要望を明確化することにより生産性に優れた新規なプロセスを提案し、事業化に結び付けることができました。
また、オキセタン樹脂の実用化においては、各種物性評価を行うことでその際立った特徴を明確化して事業化を推進するとこで、年商数億円程度の規模での工業化に成功しております。

これらの新規材料の事業化の経験は、今後の新たなる材料の創出及び事業化にも必ず役立つものと信じております。
}
%\PRLine% サブタイトル無で文字列のみの場合
\PR{研究・開発業務に役立つ幅広い知見}
{%
上記材料の特性評価および新規材料設計の過程において、粘弾性(レオロジー)測定、顕微測定による観察技術、コンピューターシミュレーション等の各種評価技術についても知見を深めてきました。
その経験を活かして基盤技術研究所において、各種「分析、観察、評価技術全般」を、所長として指導してきました。

これまでに身につけてきた幅広い知見は、新規材料を設計していく際に有効に応用できるものと確信しております。
また、これまでに経験してきた各種の科学的な議論(化学、物理、数学等)を通して、新規な研究分野、およびそれに引き続く、開発シーズを生み出していくという経験は、これまでに経験してきた分野以外の研究・開発業務に対しても十分に役立ち、その業務を推進していくことができるものと信じております。
}
\PR{総合的な研究・開発の指導}
{%
近年は、自身が身につけてきたこれまでの経験および知見を若手研究者へと伝達できるように、社内での研修、および、新化学技術推進協会での講習会も企画し、次世代を担う若手研究者に向けた教育も実施しております。

これらの幅広い経験を生かして、あらゆる世代の研究員をも巻き込んだ形で研究開発業務を推進していくことができるものと信じております。
}
\end{自己PR}
%----------------------------------------------------
% 文章の末尾を示す。右詰め。
%----------------------------------------------------
\begin{flushright}
以上。
\end{flushright}

\end{document}


%%%%%%%%%%%%%%%%%%%%%%%%%%%%%%%%%%%%%%%%%%%%%%%%%%%%
%----------------------------------------------------
% マクロ説明
%----------------------------------------------------
% \fromtoマクロ: 期間を示す文字列を生成するマクロ
%
%       形式:
%       \fromto{開始年月}{終了年月}
%
%       開始と終了の年月をYYYY/MM形式で入力すると
%       その間の期間を自動的に計算して
%       YYYY年MM月 → YYYY年MM月 (XX年YYヵ月)
%       という形式の文字列を生成する。
%       使用する場所によって縦向きか横向きかを判断し
%       自動的に適切な向きの出力をする。
%       \reverseDate{yes}マクロを使用すると、縦向きの出力の時
%       矢印が下から上を向いたものとなる。
%       これは最近の事物を上に持ってきたときに
%       有効な期間の表示となる。
%----------------------------------------------------
% \箇条書き環境: TeXのitemizeと同様な箇条書きの環境を提供する
%
%       形式:
%       \begin{箇条書き}
%               \項目 マクロ
%               :
%       \end{箇条書き}
%
%       TeXのitemizeと同じ箇条書きの環境であるが、
%       表の中で書いたときに周囲に余計なマージンをつけずにいっぱいに広がり
%       項目間も適切な間隔になるように調節したもの。
%       この環境の中では以下の\項目マクロが使える
%
%       形式:
%       \項目{項目の記述}
%
%       箇条書き環境の中でのみ使用でき、箇条書きの一行を出力する
%------------------------------------------------------
% \説明リスト環境: TeXのdescriptionと同様な説明リストの環境を提供する
%
%       形式:
%       \begin{説明リスト}
%               \説明{項目名}{項目の説明}
%               :
%       \end{説明リスト}
%
%       TeXのdescriptionと同じ説明リストの環境であるが、
%       表の中で書いたときに周囲に余計なマージンをつけずにいっぱいに広がり
%       項目間も適切な間隔になるように調節したもの。
%
%       この環境の中では以下の\説明マクロが使える
%
%       形式:
%       \説明{項目名}{項目の説明}
%
%       説明リスト環境の中でのみ使用でき、説明書きの一行を出力する
%       項目名は太字で出力され、項目の説明が一定の間隔の後に出力される。
%       表の中など出力領域が狭いときは、スペース節約のために
%       項目名が表示されないことがある。これは意図的なもので障害ではない。
%------------------------------------------------------
%
% テーブルの幅設定マクロ
% 以下の形式でリストを表形式で表示するときの幅が設定できる
% 幅の単位はTeXで許される単位
%
% pt    ポイント [1pt 〜eq 0.35mm]     in      インチ [1in = 25.4mm = 72.27pt]
% pc    パイカ [1pc=12pt]               em      現在有効な書体の文字Mの幅
% bp    big point 1bp 〜eq 1in         ex      現在有効な書体の文字xの高さ
% dd    didot point 1dd 〜eq 1.07pt    zw      現在有効な全角漢字の幅
% mm    ミリメートル [1mm 〜eq 2.85pt  zh      現在有効な全角漢字の高さ
% cm    センチメートル [1cm = 10mm]
%
% 例:\職歴リスト幅設定{8zw}{28zw}{10zw}
%
% \資格リスト幅設定{カラム1の幅}{カラム2の二段組時の幅}{カラム2の二段組でない時の幅}
% \職歴リスト幅設定{カラム1の幅}{カラム2の幅}{カラム3の幅}
% \業務リスト幅設定{カラム1の幅}{カラム2の幅}{カラム3の幅}
% \プロジェクトリスト幅設定{カラム1の幅}{カラム2の幅}{カラム3の幅}{カラム4の幅}
%
%------------------------------------------------------
