\documentclass[12pt, dvipdfmx]{beamer}

\renewcommand{\kanjifamilydefault}{\gtdefault}
%%%%%%%%%%%  package  %%%%%%%%%%%
\usepackage{bxdpx-beamer}% dvipdfmxなので必要
\usepackage{pxjahyper}% 日本語で'しおり'したい

\usepackage{amssymb,amsmath,ascmac}

\usepackage{multirow}
\usepackage{bm}

\graphicspath{{../../../Figures/gakkai/}}

\usepackage{tikz}
\usepackage{xparse}

\usepackage{multimedia}

\usetikzlibrary{shapes,arrows}
%% define fancy arrow. \tikzfancyarrow[<option>]{<text>}. ex: \tikzfancyarrow[fill=red!5]{hoge}
\tikzset{arrowstyle/.style n args={2}{inner ysep=0.1ex, inner xsep=0.5em, minimum height=2em, draw=#2, fill=black!20, font=\sffamily\bfseries, single arrow, single arrow head extend=0.4em, #1,}}
\NewDocumentCommand{\tikzfancyarrow}{O{fill=black!20} O{none}  m}{
\tikz[baseline=-0.5ex]\node [arrowstyle={#1}{#2}] {#3 \mathstrut};}

%微分関連のマクロ
%
\newcommand{\diff}{\mathrm d}
\newcommand{\difd}[2]{\dfrac{\diff #1}{\diff #2}}
\newcommand{\difp}[2]{\dfrac{\partial #1}{\partial #2}}
\newcommand{\difdd}[2]{\dfrac{\diff^2 #1}{\diff #2^2}}
\newcommand{\difpp}[2]{\dfrac{\partial^2 #1}{\partial #2^2}}

%目次スライド
\AtBeginSection[]{
  \frame{\tableofcontents[currentsection]}
}

%アペンディックスのページ番号除去
\newcommand{\backupbegin}{
   \newcounter{framenumberappendix}
   \setcounter{framenumberappendix}{\value{framenumber}}
}
\newcommand{\backupend}{
   \addtocounter{framenumberappendix}{-\value{framenumber}}
   \addtocounter{framenumber}{\value{framenumberappendix}} 
}

\newcommand{\rmd}{\mathrm{d}}
\newcommand{\dd}[1]{\dfrac{\mathrm{d} #1}{\mathrm{d} x}}

%%%%%%%%%%%  theme  %%%%%%%%%%%
\usetheme{Copenhagen}
% \usetheme{Metropolis}
% \usetheme{CambridgeUS}
% \usetheme{Berlin}

%%%%%%%%%%%  inner theme  %%%%%%%%%%%
% \useinnertheme{default}

% %%%%%%%%%%%  outer theme  %%%%%%%%%%%
\useoutertheme{default}
% \useoutertheme{infolines}

%%%%%%%%%%%  color theme  %%%%%%%%%%%
%\usecolortheme{structure}

%%%%%%%%%%%  font theme  %%%%%%%%%%%
\usefonttheme{professionalfonts}
%\usefonttheme{default}

%%%%%%%%%%%  degree of transparency  %%%%%%%%%%%
%\setbeamercovered{transparent=30}

% \setbeamertemplate{items}[default]

%%%%%%%%%%%  numbering  %%%%%%%%%%%
% \setbeamertemplate{numbered}
\setbeamertemplate{navigation symbols}{}
\setbeamertemplate{footline}[frame number]

%%%%%%%%%%%%%%%%%%%%%%%%%%%%%%%%%%%
\title
{化学系企業で物理と化学の狭間で\\考えてきたこと}
\subtitle{~コウモリ研究者の戯言~}
\author[東亞合成 佐々木]{佐々木裕}
\institute[東亞合成]{東亞合成}
\date{Nobember 26, 2021}
%%%%%%%%%%%%%%%%%%%%%%%%%%%%%%%%%%
\begin{document}
%%%%%%%%%%%%%%%%%%%%%%%%%%%%%%%%%%
\begin{frame}\frametitle{}
	\titlepage
\end{frame}
% %%%%%%%%%%%%%%%%%%%%%
% \section*{}
% %
% \begin{frame}
% %[allowframebreaks]
% {Outline}
% 	\tableofcontents
% \end{frame}

%%%%%%%%%%%%%%%%%%%%%
\section{はじめに}
%%%%%%%%%%%%%%%%%%%%%%%%%%%%%%%%%%%%%%%%%%%%%
\subsection{はじめに}
\begin{frame}
    \frametitle{はじめに}
    \begin{block}{今回のテーマ}
        「計算で物事を理解する予測する」\\
        ~産業界の実問題に立ち向かうサイエンス~

      22人の計算科学と先端実験の先駆者たちが産業界の実問題解決への手掛かりを開示します。
    \end{block}
    
    \begin{exampleblock}{私のお話}
        「理解する」という人間の行動について、フォーカス

        21人の計算に関するタイトなお話 + おまけの与太話
    \end{exampleblock}
\end{frame}


\subsection{自己紹介}

\begin{frame}
	\frametitle{自己紹介}
\end{frame}

\subsection{モデル化への私のあがき}


\section{考えてきたこと}
\subsection{化学のやり方}
    \begin{itemize}
        \item 基本的には天下りを受容
        \item 見えないものを受け入れる
        \item 
    \end{itemize}
\subsection{物理のやり方}



\subsection{抽象的?}


\begin{frame}
    \frametitle{抽象}

    「抽象」という語については、「事物や表象からある性質・共通性・本質を抽(ひ)き出して把握する」つまり「象を抽き出す」という意味を持つ語


    \begin{itemize}
        \item 個々の事物の本質・共通の属性を抜き出して、一般的な概念をとらえるさま。
        \item 単に概念的に思考されるだけで、実際の形態・内容を持たないさま。
    \end{itemize}

    後者の意味の反意語は、具体的

    Concrete, Specific

\end{frame}

\begin{frame}
    \frametitle{抽象と捨象}
    \begin{itemize}
        \item 抽き出す行為と捨てる行為
        \item 不要なものに埋もれた中から本質につながる単純化
        \item 粗視化はどちら?
        \item 熊井先生の走り回り画法
    \end{itemize}
\end{frame}


\begin{frame}
    \frametitle{<title>}

    具体的

    Concrete, Specific


    反意語として使われる

    ab-
    struct


\end{frame}
\section{おすすめ}

\subsection{私のやり方}
\begin{frame}
    \frametitle{私のやり方}
    \begin{block}{おすすめのやり方}
        \begin{itemize}
            \item 急がば回れ
            \item 備えよ常に
            \item 腑に落とす(落ちる)
        \end{itemize}
    \end{block}
\end{frame}


\end{document}
