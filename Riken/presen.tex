\documentclass[12pt, dvipdfmx]{beamer}

\renewcommand{\kanjifamilydefault}{\gtdefault}
%%%%%%%%%%%  package  %%%%%%%%%%%
\usepackage{bxdpx-beamer}% dvipdfmxなので必要
\usepackage{pxjahyper}% 日本語で'しおり'したい

\usepackage{amssymb,amsmath,ascmac}

\usepackage{multirow}
\usepackage{bm}

\graphicspath{{../Figures/simulation/}}

\usepackage{tikz}
\usepackage{xparse}

\usepackage{multimedia}

\usetikzlibrary{shapes,arrows}
%% define fancy arrow. \tikzfancyarrow[<option>]{<text>}. ex: \tikzfancyarrow[fill=red!5]{hoge}
\tikzset{arrowstyle/.style n args={2}{inner ysep=0.1ex, inner xsep=0.5em, minimum height=2em, draw=#2, fill=black!20, font=\sffamily\bfseries, single arrow, single arrow head extend=0.4em, #1,}}
\NewDocumentCommand{\tikzfancyarrow}{O{fill=black!20} O{none}  m}{
\tikz[baseline=-0.5ex]\node [arrowstyle={#1}{#2}] {#3 \mathstrut};}

%微分関連のマクロ
%
\newcommand{\diff}{\mathrm d}
\newcommand{\difd}[2]{\dfrac{\diff #1}{\diff #2}}
\newcommand{\difp}[2]{\dfrac{\partial #1}{\partial #2}}
\newcommand{\difdd}[2]{\dfrac{\diff^2 #1}{\diff #2^2}}
\newcommand{\difpp}[2]{\dfrac{\partial^2 #1}{\partial #2^2}}

%目次スライド
\AtBeginSection[]{
  \frame{\tableofcontents[currentsection]}
}

%アペンディックスのページ番号除去
\newcommand{\backupbegin}{
   \newcounter{framenumberappendix}
   \setcounter{framenumberappendix}{\value{framenumber}}
}
\newcommand{\backupend}{
   \addtocounter{framenumberappendix}{-\value{framenumber}}
   \addtocounter{framenumber}{\value{framenumberappendix}} 
}

\newcommand{\rmd}{\mathrm{d}}
\newcommand{\dd}[1]{\dfrac{\mathrm{d} #1}{\mathrm{d} x}}

%%%
\usepackage{pgfpages}
\setbeamertemplate{note page}[plain] % or [default], [compress]
\setbeameroption{show notes on second screen=right} % or bottom, ...
\newcommand{\pdfnote}[1]{\note{#1}} % as you like

%%%%%%%%%%%  theme  %%%%%%%%%%%
\usetheme{Copenhagen}
\useoutertheme{default}
%%%%%%%%%%%  font theme  %%%%%%%%%%%
\usefonttheme{professionalfonts}
%%%%%%%%%%%  numbering  %%%%%%%%%%%
% \setbeamertemplate{numbered}
\setbeamertemplate{navigation symbols}{}
\setbeamertemplate{footline}[frame number]

%%%%%%%%%%%%%%%%%%%%%%%%%%%%%%%%%%%
\title
{化学系企業で物理と化学の狭間で\\考えてきたこと}
\subtitle{~コウモリ研究者の戯言~}
\author[東亞合成 佐々木]{佐々木裕}
\institute[東亞合成]{東亞合成}
\date{Nobember 26, 2021}
%%%%%%%%%%%%%%%%%%%%%%%%%%%%%%%%%%
\begin{document}
%%%%%%%%%%%%%%%%%%%%%%%%%%%%%%%%%%
\begin{frame}\frametitle{}
	\titlepage
\end{frame}
%%%%%%%%%%%%%%%%%%%%%
\section{はじめに}
%%%%%%%%%%%%%%%%%%%%%%%%%%%%%%%%%%%%%%%%%%%%%
\subsection{はじめに}
\begin{frame}
    \frametitle{はじめに}
    \begin{block}{シンポジウムのタイトル}
        「計算で物事を理解する予測する」\\
        ~産業界の実問題に立ち向かうサイエンス~\\

      22人の計算科学と先端実験の先駆者たちが\\産業界の実問題解決への手掛かりを開示します。
    \end{block}
    
    \note[item]{本日のお題}
    \begin{exampleblock}<2->{私のお話}
        \begin{itemize}
            \item 「理解する」という人間の行動について、フォーカス
            \begin{itemize}
                \item 合成化学系出身の企業研究(開発)者が
                \item ソフトマター関連のシミュレーションを通して、
                \item 考えてきたことを紹介。
            \end{itemize}
            \item<3> 21人の計算に関するタイトなお話 + \alert{おまけの与太話}
        \end{itemize}
        \note[item]{まあ、単なる与太話ですので、お気楽にお付き合い願えればと。}
    \end{exampleblock}
\end{frame}



\subsection{自己紹介}
\begin{frame}
	\frametitle{自己紹介}
    \vspace{-3mm}
    \note[item]{私の背景を理解していただくために、簡単に自己紹介をさせていただきます。}
        \begin{columns}[T, onlytextwidth]
            \column{.48\linewidth}
                \begin{block}{大学時代}
                    \begin{itemize}
                        \item 大学で三年留年し、\\あわや放校処分
                        \item 望まぬ道の化学系へ\\(合成化学工学科)
                        \item 学部で就職できずに、修士へ
                        \item 研究の面白さに気づく
                        \begin{itemize}
                            \item ジビニルエーテルの環化重合によるクラウンエーテル類縁体
                            \item Host-Guest Chemistry
                        \end{itemize}
                    \end{itemize}
                \end{block}
            \column{.48\linewidth}
                \begin{exampleblock}{企業に就職後}
                    \begin{itemize}
                        \item 合成化学をベースとし、材料設計
                        \begin{itemize}
                            \item 経験知に基づく設計
                            \item ChemDrawの構造を無理やり材料の機能へと意味づけ
                        \end{itemize}
                        \item 留学を機会に新規材料
                        \begin{itemize}
                            \item その特性評価から、\\材料評価の道へ
                            \item 例えば、レオロジー
                        \end{itemize}
                        \item シミュレーションへと手を広げる
                    \end{itemize}
                \end{exampleblock}
        \end{columns}
\end{frame}

\subsection{モデル化への私のあがき}
\begin{frame}
    \frametitle{私の研究歴}
    \note[item]{ちょっと繰り返しになりますが、もう少し詳しく研究内容を紹介します。}
    \note[item]{海外留学時に再発見したオキセタンの評価を機会にシミュレーションの道へ}
    \begin{itemize}
        \item もともとは合成系の化学系出身
        \item カチオン重合性評価から、MOシミュレーションへ
        \item 高分子系材料一般の探索指針を求めて、メゾスケールシミュレーションへ。
    \end{itemize}
        \begin{block}{実際の内容}
            \begin{itemize}
                \item 光硬化型材料の開発において
                \begin{itemize}
                    \item 各種分子構造の試作と要求特性との相関を模索
                    % \item 光カチオン重合硬化型材料の探索
                    \item オキセタン化合物の有効性の再発見
                \end{itemize}
                \item シミュレーションをベースとしたモデル化へ
                \begin{itemize}
                    \item オキセタンの反応性について
                    \item 表面偏析のモデル化
                    \item ネットワークポリマーとネットワーク理論
                    \item フルアトムMDシミュレーションと粗視化
                \end{itemize}
            \end{itemize}
        \end{block}
\end{frame}

\begin{frame}
    \frametitle{OCTAとの出会い}
        \note[item]{OCTAの説明}
        \note[item]{JACIが開催したOCTA-WSに2004年から参加し、土井先生、川勝先生、滝本先生、および、
        三菱化学の樹神さんのような多数の民間企業の研究者の知己を得た。}
        \begin{block}{OCTAとは}
            \begin{itemize}
                \item ソフトマテリアルに対する統合的なシミュレータ
                \begin{itemize}
                    \item COGNAC、PASTA、SUSHI、MUFFIN
                    \item GOURMET というシミュレーションプラットフォーム
                \end{itemize}
            \end{itemize}
        \end{block}
        \begin{columns}[c, onlytextwidth]
            \column{.6\linewidth}
            \centering
            \includegraphics[width=\textwidth]{octa.png}
            \column{.4\linewidth}
                \begin{itemize}
                    \item マルチスケール
                    \item マルチフィジックスの統合
                    \item シームレス\\ズーミング
                \end{itemize}
        \end{columns}    
\end{frame}

\begin{frame}
    \frametitle{統合的な理解を目指して}
    \begin{itemize}
        \item マルチスケールな取り扱いで階層的な構造をイメージ
        \begin{itemize}
            \item メゾスケールの重要性
            \begin{itemize}
                \item ローカルには、自由エネルギーを最小化
                \item ローカルの微視的状態の個数倍 $\neq$ グローバル
            \end{itemize}
            \item 実事象では、平衡状態には至らない
            \begin{itemize}
                \item 時間遷移の過程で準安定状態でトラップ
                \item 生物学での、ホメオスタシス(恒常性)
                \item 自己組織化の理解
                % \item 分散システムでの自己安定化:フォールトトレラント
            \end{itemize}
        \end{itemize}
        \item マルチフィジックスは、人間の勝手な都合
        \begin{itemize}
            \item 自然はあるがままに捉えるべき
            \item 階層ごとの切り分けは無意味な場合も多い
        \end{itemize}
        \item シームレスズーミングは幻想
    \end{itemize}
    \large{\alert{実事象の統合的な理解は一筋縄では行かない!}}
\end{frame}

\begin{frame}
    \frametitle{自己組織化という概念}
    \begin{block}{自己組織化という概念}
        \begin{itemize}
            \item 材料開発でのナノテクノロジーという文脈で注目
            \item ボトムアップ式のナノテクノロジー
            \item 工学的には、自己集合体とは区別しない事が多い
        \end{itemize}
    \end{block}
    \note[item]{例えば、自己組織化という概念を例に考えてみると、}
    \begin{exampleblock}{自己集合体(平衡条件近傍で形成)とは区別する立場}
        \begin{columns}[T, onlytextwidth]
            \column{.48\linewidth}
            \begin{alertblock}{プリゴジンの散逸構造}
                \begin{itemize}
                    \item 非平衡開放系において
                    \item 平衡構造の不安定化
                    \item 自発的に形成された\\秩序構造
                \end{itemize}
            \end{alertblock}
            \column{.48\linewidth}
            \begin{alertblock}{J.M. Lehn の主張}
                \begin{itemize}
                    \item 超分子科学の提唱者
                    \item 分子自身の分子情報に従って、機能を有する組織を形成
                    % \item 情報の有無に注目
                \end{itemize}
            \end{alertblock}
        \end{columns}
    \end{exampleblock}
\end{frame}

\section{考えてきたこと}
\subsection{モデル化による現象の理解}
\begin{frame}
    \frametitle{モデル化による現象の理解}
    \note[item]{あくまでも、私にとってなんですが}
    \begin{alertblock}{化学系の人間としての過去の自身に欠けていたもの}
        物理系では当たり前の考え方\\
        「自然現象の背後にあるユニバーサリティーの理解と、\\適正なレベルでのモデル化」
    \end{alertblock}

    \begin{block}{化学系企業でありがちな状態}
        \begin{itemize}
            \item 教科書的なものの背後にある物理的、数学的な思想を理解することからの逃避
            \item 数式や物理モデルの盲目的な受認によるデータの処理
            \item 統計的な妥当性の確認の放棄
            \item 客観的な視点に基づく独立事象と従属事象の切り分けの放棄
        \end{itemize}
    \end{block}
\end{frame}

\subsection{化学と物理}
\begin{frame}
    \frametitle{化学と物理を比べると}
    \begin{columns}[T, onlytextwidth]
        \column{.48\linewidth}
        \begin{block}{化学のやり方}
            \begin{itemize}
                \item 基本的には天下りを受容
                \begin{itemize}
                    \item 見えないものを受け入れる
                    \item 有り様を実感できない分子、原子
                    \item 「誰が原子を見たか?」
                \end{itemize}
                \item 多様性を容認
                \note[item]{ちょっとイチャモンかもしれませんが、偏微分でビビる人が多いのかもしれません}
                \item 熱力学が理解できていない人が、けっこう多い
            \end{itemize}
        \end{block}
        
        \column{.48\linewidth}
        \begin{exampleblock}{物理的な考え方}
            \begin{itemize}
                \item 事象に内在する一般性
                \item その本質に迫るために\\モデル化
                \item 興味深い考え方、\\
                揺動散逸定理、中心極限定理、線形応答理論、乱雑位相近似、臨界現象(転移における普遍性)、緩和挙動、スケーリング則
            \end{itemize}
            \note[item]{これは、エイヤと私が感心したものを並べただけですが。}
        \end{exampleblock}
    \end{columns}
\end{frame}

\begin{frame}
    \frametitle{ソフトマターでの印象派物理}
    \begin{itemize}
        \item 印象派とは数学的な詳細をあえて大胆に無視することでシンプルに捉え、本質に迫るスタイルのこと
        \item 写実主義的物理との対比
        \item P-G de Gennes からのフランスでの潮流
        \item まわりの実験家とつねに対話をして、身の回りの事象に対して強い好奇心を持ち、斬新なアイデアを創出
        \item 大胆な発想での理解;「大いなる同一視」
    \end{itemize}
\end{frame}

\subsection{抽象的と具体的}
\begin{frame}
    \frametitle{抽象的に考える}
    抽象的ということを、非現実的と捉え、「えそらごと」と読んでしまう人のなんと多いことか。

    \begin{exampleblock}{抽象とは}
        「抽象」という語については、「事物や表象からある性質・共通性・本質を抽(ひ)き出して把握する」つまり「象を抽き出す」という意味を持つ語
        \begin{itemize}
            \item 個々の事物の本質・共通の属性を抜き出して、一般的な概念をとらえるさま。
            \item 単に概念的に思考されるだけで、実際の形態・内容を持たないさま。
        \end{itemize}
        後者の意味の反意語は、具体的
    \end{exampleblock}
    \large{\alert{抽象化は、モデル化に必須。}}
\end{frame}

\begin{frame}
    \frametitle{最近の風潮}
    \begin{columns}[T, onlytextwidth]
        \column{.48\linewidth}
            \begin{block}{最近の風潮}
                \begin{itemize}
                    \item 実事象はあまりに複雑で因果関係がわかりにくい。
                    \item それにも関わらず、すぐに成果を求める。
                    \item シミュレーションに対しても、考え方の指針ではなく、答えを求める。
                    \item そのようなアプローチは、汎用性を生み出さない。
                \end{itemize}
            \end{block}
        \column{.48\linewidth}
            \begin{exampleblock}{抽象と捨象}
                \begin{itemize}
                    \item 捨象は捨てる行為に、\\フォーカス
                    \item 抽き出す行為と捨てる行為
                    \item どちらも不要なものに埋もれた中から本質につながる単純化
                    \item 粗視化はどちらであるべきか?
                    \item 熊井先生の走り回り画法
                \end{itemize}
            \end{exampleblock}
    \end{columns} 
\end{frame}

\section{私のおすすめ}
\subsection{あるべき状態}
\begin{frame}
    \frametitle{MI への違和感}
    \begin{block}{機械学習について}
        機械学習は特定の分野では非常に有効
        \begin{itemize}
            \item 回帰的手法をベースとした多変量解析
            \item 自動運転のようなフィードバック系
        \end{itemize}
    \end{block}
    \begin{exampleblock}{MI への違和感}
        \note[item]{悪口ばかりでも何なんで、使い方次第とは思います。}
        \begin{itemize}
            \item (一部の方に見られる)思考を放棄したような無手勝流
            \item 少なくとも、MI を打ち出の小槌と捉えてはいけない。
            \item シミュレーションを実験の代替とする方法論は有効。
            \item 考えるための道具として有効活用すべき。
        \end{itemize}
    \end{exampleblock}
\end{frame}

\begin{frame}
    \frametitle{あるべき状態}
    \begin{exampleblock}{化学系研究者の立ち位置}
        \begin{itemize}
            \item 試行錯誤ベースで実際に物質を合成することは必須
            \item 物理側からの理論的な成果を盲目的に受け入れては駄目
        \end{itemize}
    \end{exampleblock}
    \begin{alertblock}{あるべき状態}
        \begin{itemize}
            \item 物理的な思考による事象の成り立ちの理解、および、モデル化への道すじを共有
            \item 目的を明確にし、適切な次元、スケール及び時間軸で、議論を行う
            \item 物質の多様性を前提とした化学的な方法論の整理と、適正なモデル化への挑戦
            \item 物理及び化学双方の方法論についての相互理解の深化
        \end{itemize}
    \end{alertblock}
\end{frame}

\begin{frame}
    \frametitle{基礎知識の汎用化について}
        \begin{exampleblock}{データサイエンスの企業での使いこなし}
            \begin{itemize}
                \item データサイエンティストの中途採用
                \begin{itemize}
                    \item マネージメントの難しさ $\Rightarrow$ プロの持ち腐れ
                    \item 現役データサイエンティストの満足度は低い
                    \begin{itemize}
                        \item 手本がない
                        \item 周りの理解がない
                        \item スキルアップの時間がない
                    \end{itemize}
                \end{itemize}
            \end{itemize}
        \end{exampleblock}
        \begin{alertblock}{「データサイエンスの民主化」}
            \begin{itemize}
                \item 文系、数学苦手は関係ない
                \item データをもとに客観的に考えるという基本的な概念
                \item 関係者みんなに広く浅く(深いに越したことはない)
            \end{itemize}
        \end{alertblock}
        \centering
        \Large{\alert{研究一般についても大事}}
\end{frame}

\subsection{私のやり方}
\begin{frame}
    \frametitle{私のやり方}
        \begin{itemize}
            \LARGE
            \item 急がば回れ
            \begin{itemize}
                \Large
                \item<2-> 慌ててやっても無駄
                \item<2-> キチンと組み立てないと無駄
            \end{itemize}
            \item 備えよ常に
            \begin{itemize}
                \Large
                \item<3-> 見えないものにも前もって
                \item<3-> 泥縄にならないように
            \end{itemize}
            \item 腑に落とす(落ちる)
            \begin{itemize}
                \Large
                \item<4> 消化して使いこなす
                \item<4> 頭でっかちにならない
            \end{itemize}
        \end{itemize}
\end{frame}

% \begin{frame}
%     \frametitle{概念の理解}
%         説明変数と目的変数との関係をモデル化
%         \begin{itemize}
%             \item 例えば、ランダムフォレスト
%             \begin{itemize}
%                 \item 説明変数の選択への制約が少ない。
%                 \item 過学習を影響を排除しやすい。
%             \end{itemize}
%             \item 
            
%         \end{itemize}
% \end{frame}

\subsection{自分の頭で考える}
\begin{frame}
    \frametitle{他人の意見について}
        \begin{columns}[c, onlytextwidth]
            \column{.54\linewidth}
            \large
            \begin{block}{その道のプロの言うこと}
                \begin{itemize}
                    \large
                    \item それなりの確からしさ
                    \item 前提条件の確認が必要
                    \begin{itemize}
                        \large
                        \item 常識が異なる
                        \item 暗黙の了解が多数
                    \end{itemize}
                    \item 素人が下手に使う怖さ
                \end{itemize}
            \end{block}
            \column{.02\linewidth}

            sikenn

            % \large{$\Rightarrow$}
            
            \column{.44\linewidth}<2>
            \LARGE
                \alert{「盲目的に\\信じてはだめ」}
        \end{columns}
\end{frame}

\begin{frame}
    \frametitle{自分の頭で考える}
        \begin{alertblock}{胃の腑に落とすということは?}
            無理やり胃に落としてもだめ!!

            \begin{columns}[T, onlytextwidth]
                \column{.48\linewidth}
                
                \begin{block}{咀嚼するための基礎学力}
                    STEAM 
                    \begin{itemize}
                        \item Science
                        \item Technology
                        \item Engineering
                        \item \alert{Art}
                        \begin{itemize}
                            % \item<2> 体系立てて捉える
                            \item<2-> 成り立ちの美しさ
                            \item<2-> 哲学的な統一性 
                        \end{itemize}
                        \item Mathematics
                    \end{itemize}
                \end{block}
                \column{.48\linewidth}
                \begin{exampleblock}<3>{消化(使いこなす)ために?}
                    \begin{itemize}
                        \item 特定分野に囚われない広範な知見
                        \item 自由な議論
                        \item 締め切りを決めない
                        \item ゆっくり考える
                        \item 数値化にこだわらない
                        \item 目に見えないものを\\大事に
                    \end{itemize}
                \end{exampleblock}
            \end{columns}
        \end{alertblock}
\end{frame}

\begin{frame}
    \frametitle{まとめに代えて}
        \begin{exampleblock}{私のアプローチ}
            \begin{itemize}
                \item 自由に議論できる場の創設
                \begin{itemize}
                    \item Slack を利用して、「東海ソフトマター」という\\議論の場を設置
                    \item 大学、企業半々程度の参加者
                    \item それをベースに「ザツダン会」を開催
                \end{itemize}
                \item 基本的な知見の再整理
                \begin{itemize}
                    \item Moodle システムを利用して、LMSサイトを整備中
                    \item 自身の初心者としての疑問点にフォーカスして整理
                    \item 対象:レオロジー、高分子物理、統計等
                \end{itemize}
            \end{itemize}
        \end{exampleblock}
\end{frame}
\end{document}
