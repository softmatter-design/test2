\documentclass[uplatex,a4paper, 11pt]{jsarticle}


% % 数式
% \usepackage{amsmath,amsthm,amssymb}
% \usepackage{bm}
% % 画像
% \usepackage[dvipdfmx]{graphicx, color}

% \usepackage{multirow}
% \usepackage{wrapfig}
\usepackage{ascmac}

% \graphicspath{{../../_Figures//}{../../_Figures/PhaseSeparation/}}


% %
% %数式の途中で改行
% \allowdisplaybreaks[3]
% %
% %微分関連のマクロ
% %
% \newcommand{\diff}{\mathrm d}
% \newcommand{\difd}[2]{\dfrac{\diff #1}{\diff #2}}
% \newcommand{\difp}[2]{\dfrac{\partial #1}{\partial #2}}
% \newcommand{\difdd}[2]{\dfrac{\diff^2 #1}{\diff #2^2}}
% \newcommand{\difpp}[2]{\dfrac{\partial^2 #1}{\partial #2^2}}

% ページ設定
\pagestyle{empty}
% 高さの設定
\usepackage[top=20truemm,bottom=20truemm,left=20truemm,right=20truemm]{geometry}
% \setlength{\textheight}{\paperheight}   % ひとまず紙面を本文領域に
% \setlength{\topmargin}{-5.4truemm}      % 上の余白を20mm(=1inch-5.4mm)に
% \addtolength{\topmargin}{-\headheight}  % 
% \addtolength{\topmargin}{-\headsep}     % ヘッダの分だけ本文領域を移動させる
% \addtolength{\textheight}{-40truemm}    % 下の余白も20mmに%% 幅の設定
% \setlength{\textwidth}{\paperwidth}     % ひとまず紙面を本文領域に
% \setlength{\oddsidemargin}{-5.4truemm}  % 左の余白を20mm(=1inch-5.4mm)に
% \setlength{\evensidemargin}{-5.4truemm} % 
% \addtolength{\textwidth}{-40truemm}     % 右の余白も20mmに

% \setpagelayout{margin=2cm}

% \abovecaptionskip=-5pt
% \belowcaptionskip=-5pt
%
\renewcommand{\baselinestretch}{1.0} % 全体の行間調整
%\renewcommand{\figurename}{Fig.}
%\renewcommand{\tablename}{Tab.}
%

% \makeatletter 
% \def\section{\@startsection {section}{1}{\z@}{1.5 ex plus 2ex minus -.2ex}{0.5 ex plus .2ex}{\large\bf}}
% \def\subsection{\@startsection{subsection}{2}{\z@}{0.2\Cvs \@plus.5\Cdp \@minus.2\Cdp}{0.1\Cvs \@plus.3\Cdp}{\reset@font\normalsize\bfseries}}
% \makeatother 

%
% \usepackage[dvipdfmx]{hyperref}
% \usepackage{pxjahyper}
% \hypersetup{%
% bookmarksnumbered=true,%
% colorlinks=true,%
% setpagesize=false,%
% pdftitle={タイトル},%
% pdfauthor={佐々木裕},%
% pdfsubject={},%
% pdfkeywords={キーワード}}

\title{化学系企業で物理と化学の狭間で考えてきたこと\\ \Large{~コウモリ研究者の戯言~}}
% \subtitle{~コウモリ研究者の戯言~}
\author{佐々木裕\thanks{東亞合成株式会社}}
\date{}%{2021, Nov. 25-26}

\begin{document}

\maketitle
\vspace{-3mm}
% \section*{概要}

近年、産業界のみならず学術界においても日本の凋落が顕著になってきていると感じている方は、少なからずいらっしゃるでしょう。
また、東北大震災、各地の土砂災害、COVID-19 等々の災厄に対しても、後手に回った対応の不手際の多さには目に余るものがあります。
% 個人ベースの指導者の問題とするのであれば、それは、日本に限らず、世界中でポピュリストの台頭が憂うべき問題として議論されています。
一方、AI 関連技術としての機械学習やシミュレーションのベースとなる計算機資源の拡充や方法論の革新的な進展が、
世界的な規模で加速度的に進展していることは言を俟ちません。
日本のシステム的な疲弊を放置すれば、先進国からの遅れは取り返しのつかないものへと広がってしまいます。

規模の小さい化学系企業に所属する一介の研究者の立場から、国家のあるべき姿のような大所高所に立った俯瞰的な意見を述べるようなことは
できませんが、上記のような周回遅れは材料開発の現場においても顕著に生じていると痛感しています。

話者は、大学時代の合成化学の知見をベースにして化学系企業での研究・開発生活をスタートし、
必要に迫られて学んできた評価技術の背景にある物理的な思考の重要性に気づき、
そこから、レオロジーやシミュレーションのベースとなるソフトマター物理へと傾倒してきました。
その過程において幅広い分野の研究者の方の薫陶を受けることにより、化学系の人間としての過去の自身に欠けていたものが、
自然現象の背後にあるユニバーサリティーの理解と適正なレベルでのモデル化だと感じてきました。

近年の CAX(Computer Aided something) の長足の進歩を実際の材料の開発へとつなげるためには、
化学系の研究者に依る試行錯誤に基づく実証的なアプローチを欠かすことはできません。
そして、この実材料の創成の過程において、
化学系の研究者が物理側からの理論的な成果を盲目的に受け入れるだけでなくその背景を理解し、
また、モデルを創造する物理の人たちにも物質の中に内在する多様性を実感していただくことで条件設定を適正化するような、
そのような相互理解を深めていくことこそが重要なのだと考えるに至りました。
\vspace{3mm}
\begin{boxnote}
    ここでお話させていただく内容を、以下に簡単にまとめました。
\begin{enumerate}
    \item かんたんな自己紹介
    \begin{itemize}
        \item 研究・開発歴
        \item アプローチ方法の変遷
        \item 私のモットー
            % \begin{itemize}
            %     \item 急がば回れ
            %     \item 備えよ常に
            % \end{itemize}
    \end{itemize}
    \item (少なくとも化学系企業で)開発にありがちな状態
    \begin{itemize}
        \item 教科書的なものの背後にある物理的、数学的な思想の理解からの逃避
        \item 数式や物理モデルの盲目的な受認によるデータの処理
        \item 客観的な視点に基づく独立事象と従属事象の切り分けの放棄
    \end{itemize}
    \item あるべき状態
    \begin{itemize}
        \item 物理的なアプローチによる成り立ちの解明およびモデルとしての落とし込み
        \item 多様性を前提とした化学的な試行錯誤に基づく物質の創成
        \item 両者の止揚
    \end{itemize}
    % \item 物理化学として物質を見直すと
    % \begin{itemize}
    %     \item 物質の三態について
    %     \item 流れるということは?
    %     \item 応力の由来は?
    % \end{itemize}
    % \item 粘弾性の基礎
    % \begin{itemize}
    %     \item 粘性と弾性についての再確認
    %     \item 粘弾性のモデル化
    %     \item 少しだけ実事象に近づけると
    % \end{itemize}
    \item まとめに代えてのディスカッション
\end{enumerate}
\end{boxnote}




% 物事には仕組みがあることを忘れない。

% 切り取りではなく全体を見通すこと

% 仕組みを作り出すには論理の流れが必要

% 先達の仕事には理屈の裏付けがある

% ときには、それは後付のときもあるが。

% QCの功罪

% よくあるまずいこと

% 積んだブロックを壊すことなく無理やり上に積み上げる。

% 急いで結果を出そうとする

% 論理は再構築することで磨かれる。

% 自然の多様性とその裏にある類似性

% ソフトマター物理でよく見られる「大いなる同一視」

% 物理でよくあるユニバーサリティーへの期待

\end{document}