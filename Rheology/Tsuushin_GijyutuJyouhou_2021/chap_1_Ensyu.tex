\documentclass[uplatex,dvipdfmx,a4paper,11pt]{jsarticle}

\usepackage{docmute}


% 数式
\usepackage{amsmath,amsthm,amssymb}
\usepackage{bm}
% 画像
\usepackage{graphicx}

\usepackage{multirow}
\usepackage{wrapfig}
\usepackage{ascmac}
\usepackage{xcolor}


\usepackage{makeidx}
\makeindex

\graphicspath{{../../_Figures//}{../../_Figures/Rheology/}}

\usepackage{qrcode}
\setlength\lineskiplimit{0pt}
\setlength\normallineskiplimit{0pt}

\usepackage{qexam}

\usepackage{titlesec}
\titleformat*{\section}{\Large\bfseries}
\titleformat*{\subsection}{\large\bfseries}
\titleformat*{\subsubsection}{\normalsize\bfseries}
\titleformat*{\paragraph}{\normalsize\bfseries}

% ページ設定
% \pagestyle{empty}
% 高さの設定
\setlength{\textheight}{\paperheight}   % ひとまず紙面を本文領域に
\setlength{\topmargin}{-5.4truemm}      % 上の余白を20mm(=1inch-5.4mm)に
\addtolength{\topmargin}{-\headheight}  % 
\addtolength{\topmargin}{-\headsep}     % ヘッダの分だけ本文領域を移動させる
\addtolength{\textheight}{-40truemm}    % 下の余白も20mmに%% 幅の設定
\setlength{\textwidth}{\paperwidth}     % ひとまず紙面を本文領域に
\setlength{\oddsidemargin}{-5.4truemm}  % 左の余白を20mm(=1inch-5.4mm)に
\setlength{\evensidemargin}{-5.4truemm} % 
\addtolength{\textwidth}{-40truemm}     % 右の余白も20mmに
% 図と本文との間
%\abovecaptionskip=-5pt
%\belowcaptionskip=-5pt
%
% 全体の行間調整
% \renewcommand{\baselinestretch}{1.0} 
% 図と表
%\renewcommand{\figurename}{Fig.}
%\renewcommand{\tablename}{Tab.}
%

% \makeatletter 
% \def\section{\@startsection {section}{1}{\z@}{1.5 ex plus 2ex minus -.2ex}{0.5 ex plus .2ex}{\large\bf}}
% \def\subsection{\@startsection{subsection}{2}{\z@}{0.2\Cvs \@plus.5\Cdp \@minus.2\Cdp}{0.1\Cvs \@plus.3\Cdp}{\reset@font\normalsize\bfseries}}
% \makeatother 

\usepackage[dvipdfmx,%
 bookmarks=true,%
 bookmarksnumbered=true,%
 colorlinks=false,%
 setpagesize=false,%
 pdftitle={数式に頼らない直感的理解による材料設計のためのレオロジー⼊⾨},%
 pdfauthor={佐々木裕},%
 pdfsubject={},%
 pdfkeywords={レオロジー; 材料設計; }]{hyperref}
\usepackage{pxjahyper}

\usepackage{plext}

\usepackage{niceframe} 
\usepackage{framed}
\newenvironment{longartdeco}{%
  \def\FrameCommand{\fboxsep=\FrameSep \artdecoframe}%
  \MakeFramed {\FrameRestore}}%
 {\endMakeFramed}
 
\usepackage{siunitx}

\newcommand{\rmd}{\mathrm{d}}

\usepackage[inline]{showlabels}

\begin{document}

\question{演習問題 1}
内容を振り返るために、以下に示した文章例の中から適切な記述のものを複数選んでください。
\begin{qlist}
	\qitem レオロジーとはどのようなものでしょうか?
		\begin{qlist2}
			\qitem レオロジーとは「物質の変形と流動に関する科学」であり、これまでの物理理論とはかけ離れた新規な測定技術です。
			\qitem レオロジーのベースとなる物体の変形や流動に関する物理学は、古くから弾性論や流体論として存在していました。
			\qitem 弾性論はフックの法則を基本にして弾性固体の力学的な性質を記述し、流体論はニュートンの法則により粘性を持つ流体の流れ方の挙動を明確にします。
			\qitem レオロジー測定は、物質の変形や流動を測って、そこに加えられた刺激を推定する逆問題を解きます。
			\qitem レオロジー測定を通して、物質の持つ各種の特性を比較することができるようになります。
    \end{qlist2}
    \vspace{3mm}
	\qitem 会社の仕事とレオロジーとの関係について確認してみましょう。
		\begin{qlist2}
			\qitem レオロジーという学問は高度に学際的な科学であり、同時に、それぞれの要素技術が大きく異る多岐にわたる対象を多様な切り口で議論していきます。
			\qitem レオロジーは、物質の特性を絶対値として測定する方法で、物質に含有される成分を定量的に分析することができます。
			\qitem 商品の開発の「原料から材料そして商品へと仕立てる過程」において、特性の評価にレオロジーを活用することができます。
            \qitem 例えば、人間の心地良さの定量化や、原料、材料の機能設計にレオロジーが活用されている事例が多くあります。
            \qitem 人間の感覚のような抽象的なものは定量化する方法がありませんから、できるだけ過去の記憶や勘を働かせることで設計を進めるしかありません。
    \end{qlist2}
    \vspace{3mm}
	\qitem 人がもっている感覚とレオロジーとはどのような関係があるのでしょうか?
		\begin{qlist2}
            \qitem オノマトペとは様々な状態や感情を言葉に表したものであり、レオロジー的な違いを定性的に表現できます。
            \qitem 人間は身の回りにある様々な状態や感情に言葉を与えますが、そのような定性的な評価はレオロジーとは関係ありません。
			\qitem 人間は、ものを叩いたり触ったりして刺激を与えて、その応答を音で聞いたり手触りで感じることで、レオロジー評価を行う事ができます。
			\qitem 人間のレオロジー的な感覚を表現したウェーバー・フェヒナーの法則があり、基準とする値に応じて違いを見極める閾値が変わります。
            \qitem 人間の感覚は強い刺激の変化にとても敏感で、絶対的な差を感じることができます。
    \end{qlist2}
    \vspace{3mm}
	\qitem レオロジーを理解するために考えるべきことは?
		\begin{qlist2}
			\qitem 人間は手触りでレオロジー的な違いを判断できますが、定量的な評価を判断することはあまりうまくできません。
            \qitem 感覚的な違いを言葉で表せば、十分に人に伝わる定量的な評価になります。
            \qitem 以前に実施した実験の結果は実験事実なのだから、その内容の整合性など考えることなく過去の知見として使っていけます。
			\qitem 著者のおすすめは、標語的には「急がば回れ」であり、慌てずに、イメージとして全体像をザックリと捕まえることができれば、理解は一気に容易になると期待しています。
			\qitem 会社の仕事として命令されたような事項であっても、「何のためにやりたいのか」という目的と「何をやりたいのか」という目標をきちんと設定することが最も大事になります。
		\end{qlist2}
\end{qlist}

\question{演習問題 2}
内容を振り返るために、テキストで用いた言葉を使って簡単な穴埋めを行ってください。

\begin{qparts}
    \qpart 「レオロジーとは」について、以下の\qbox{(a)}から\qbox{(i)}までのカッコを埋めてください。
    \begin{qlist}
      \qitem レオロジーとは「物質の\qbox{}に関する科学」であり、既存の弾性論や流体論をベースとして、物質の持つ各種の特性を比較することができる学問です。
      決して新規な\qbox{}ではなく、過去の\qbox{}を利用していく必要があります。
      \qitem レオロジーの対象は多岐にわたりますが、基本的に物質の特性を\qbox{}する技術であり、\qbox{}が主となります。
      \qitem 人間のレオロジー的な能力はとても高く、様々な違いを定性的に感じることができます。ただし、その\qbox{}に感じるだけであり、また、基準に応じて\qbox{}することを理解する必要があります。
      \qitem レオロジーが相対的な比較であることをきちんと理解して、落ち着いて\qbox{}で議論しましょう。また、\qbox{}を明確にすることはとても大事です。
    \end{qlist}

    \begin{itembox}[l]{選択肢}
      \begin{center}
        \begin{tabular}{lllll}
                1. 目的と目標&2. 差を相対的に&3. 相対的に比較&4. 定性的な評価 & 5. 閾値が変化 \\
                6. 変形と流動&7. 多様な知見&8. わかり易い言葉 & 9. 測定技術
        \end{tabular}
      \end{center}
    \end{itembox}
\end{qparts}

\question{演習問題 3}
\begin{qparts}
    \qpart 「何をなんのためにやりたいのか」という目的や目標は皆さんそれぞれのものをお持ちであり、そのためにレオロジーを学ばれているのだと思います。
    折角の機会ですから、一度ご自身の有りたい姿について考えをまとめてみてはいかがでしょうか。

    以下に、かんたんに設問の形で項目を上げてみました。
    \vspace{-2mm}
    \begin{qlist}
        \qitem ご自分の仕事を、全く事前の知識を持たない他の人にでも理解できるように説明してみましょう。
        \qitem その仕事の中での、ご自分の役割をシンプルに表現してみてください。
        \qitem ご自分の役割の中で、何をやらなくてはいけないと感じていますか?
        \qitem その使命は、何のためにやっているのでしょうか?
        \qitem 上述の状態の中で、レオロジーにはどのようなことを期待していますか?
    \end{qlist}
    \vspace{-2mm}
    \qpart もしもご相談されたいことがあれば、書いていただければ相談に乗ります。(なお、こちらは提出は義務ではありません。)
\end{qparts}

\end{document}