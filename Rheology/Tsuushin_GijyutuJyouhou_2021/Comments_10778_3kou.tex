\documentclass[uplatex,dvipdfmx,a4paper,11pt]{jsreport}

\usepackage{docmute}


% 数式
\usepackage{amsmath,amsthm,amssymb}
\usepackage{bm}
% 画像
\usepackage{graphicx}

\usepackage{multirow}
\usepackage{wrapfig}
\usepackage{ascmac}
\usepackage{xcolor}


\usepackage{makeidx}
\makeindex

\graphicspath{{../../_Figures//}{../../_Figures/Rheology/}}

\usepackage{qrcode}
\setlength\lineskiplimit{0pt}
\setlength\normallineskiplimit{0pt}

\usepackage{qexam}

\usepackage{titlesec}
\titleformat*{\section}{\Large\bfseries}
\titleformat*{\subsection}{\large\bfseries}
\titleformat*{\subsubsection}{\normalsize\bfseries}
\titleformat*{\paragraph}{\normalsize\bfseries}

% ページ設定
% \pagestyle{empty}
% 高さの設定
\setlength{\textheight}{\paperheight}   % ひとまず紙面を本文領域に
\setlength{\topmargin}{-5.4truemm}      % 上の余白を20mm(=1inch-5.4mm)に
\addtolength{\topmargin}{-\headheight}  % 
\addtolength{\topmargin}{-\headsep}     % ヘッダの分だけ本文領域を移動させる
\addtolength{\textheight}{-40truemm}    % 下の余白も20mmに%% 幅の設定
\setlength{\textwidth}{\paperwidth}     % ひとまず紙面を本文領域に
\setlength{\oddsidemargin}{-5.4truemm}  % 左の余白を20mm(=1inch-5.4mm)に
\setlength{\evensidemargin}{-5.4truemm} % 
\addtolength{\textwidth}{-40truemm}     % 右の余白も20mmに
% 図と本文との間
%\abovecaptionskip=-5pt
%\belowcaptionskip=-5pt
%
% 全体の行間調整
% \renewcommand{\baselinestretch}{1.0} 
% 図と表
%\renewcommand{\figurename}{Fig.}
%\renewcommand{\tablename}{Tab.}
%

% \makeatletter 
% \def\section{\@startsection {section}{1}{\z@}{1.5 ex plus 2ex minus -.2ex}{0.5 ex plus .2ex}{\large\bf}}
% \def\subsection{\@startsection{subsection}{2}{\z@}{0.2\Cvs \@plus.5\Cdp \@minus.2\Cdp}{0.1\Cvs \@plus.3\Cdp}{\reset@font\normalsize\bfseries}}
% \makeatother 

\usepackage[dvipdfmx,%
 bookmarks=true,%
 bookmarksnumbered=true,%
 colorlinks=false,%
 setpagesize=false,%
 pdftitle={数式に頼らない直感的理解による材料設計のためのレオロジー⼊⾨},%
 pdfauthor={佐々木裕},%
 pdfsubject={},%
 pdfkeywords={レオロジー; 材料設計; }]{hyperref}
\usepackage{pxjahyper}

\usepackage{plext}

\usepackage{niceframe} 
\usepackage{framed}
\newenvironment{longartdeco}{%
  \def\FrameCommand{\fboxsep=\FrameSep \artdecoframe}%
  \MakeFramed {\FrameRestore}}%
 {\endMakeFramed}
 
\usepackage{siunitx}

\newcommand{\rmd}{\mathrm{d}}

\pagestyle{empty}

\begin{document}

\section*{全体を通してのコメント}
\subsection*{第一章}
\subsubsection*{演習問題1、2について}
ほぼ、理解されているようですので、特に問題はありません。

\subsubsection*{演習問題3について}

皆さん、ご自身の中で、緩和現象についてのイメージをきちんと持たれているようです。
ただ、まずは、ご自身の手でも理解できるマイクロスケールでの緩和現象として捉えることから始めて、
その原因がミクロな流動であると、切り分けたほうがいいかも知れません。

\subsection*{第二章}
\subsubsection*{演習問題1、2について}
ほぼ、理解されているようですので、特に問題はありません。



\subsubsection*{演習問題3について}

皆さん、それぞれのご興味の対象を持たれており、それを言葉で説明できています。
全く問題ありません。



\clearpage

\section*{江森 様}
\subsection*{第一章}
\subsubsection*{演習問題1、2について}
理解されているようですので、特に問題はありません。

\subsubsection*{演習問題3について}

ご自身の中で、緩和現象についてのイメージをきちんと持たれているようです。
ただ、緩和時間そのものは、弾性と粘性の度合いを決めるものではありません。
観測時間との比であるデボラ数により、粘性的であったり、弾性的であったりすると理解したほうがいいと思います。


\subsection*{第二章}
\subsubsection*{演習問題1、2について}
理解されているようですので、特に問題はありません。

\subsubsection*{演習問題3について}

非ニュートン性を示すメレンゲについて記述されたのですね。
泡が安定的に生じて構造を形成するので、非ニュートン性が発言するのですね。

ただ、私は、卵白がメレンゲへと変化する過程をシア・シックニングと捉えるべきかどうかについてはよくわかりません。
まあ、撹拌により、見かけの粘度が上昇するといえば、確かにそうなのですが、構造の変化が極端な気もします。

\clearpage
\section*{平井 様}
\subsection*{第一章}
\subsubsection*{演習問題1、2について}
ほぼ、理解されているようですので、特に問題はありません。

\subsubsection*{演習問題3について}

ご自身の中で、緩和現象についてのイメージをきちんと持たれているようです。

\subsection*{第二章}
\subsubsection*{演習問題1、2について}
ほぼ、理解されているようですので、特に問題はありません。


\subsubsection*{演習問題3について}

ご自身の中で、きちんとイメージを持たれているようです。全く問題ありません。

ただ、食品レオロジーはなかなかにややこしくて、小麦粉や米粉に水が回るという現象は、ミクロには複雑なことが生じているような気がします。


\clearpage

\end{document}