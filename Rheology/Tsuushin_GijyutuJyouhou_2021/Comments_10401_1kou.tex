\documentclass[uplatex,dvipdfmx,a4paper,11pt]{jsreport}

\usepackage{docmute}


% 数式
\usepackage{amsmath,amsthm,amssymb}
\usepackage{bm}
% 画像
\usepackage{graphicx}

\usepackage{multirow}
\usepackage{wrapfig}
\usepackage{ascmac}
\usepackage{xcolor}


\usepackage{makeidx}
\makeindex

\graphicspath{{../../Figures//}{../../Figures/Rheology/}}

\usepackage{qrcode}
\setlength\lineskiplimit{0pt}
\setlength\normallineskiplimit{0pt}

\usepackage{qexam}

\usepackage{titlesec}
\titleformat*{\section}{\Large\bfseries}
\titleformat*{\subsection}{\large\bfseries}
\titleformat*{\subsubsection}{\normalsize\bfseries}
\titleformat*{\paragraph}{\normalsize\bfseries}

% ページ設定
% \pagestyle{empty}
% 高さの設定
\setlength{\textheight}{\paperheight}   % ひとまず紙面を本文領域に
\setlength{\topmargin}{-5.4truemm}      % 上の余白を20mm(=1inch-5.4mm)に
\addtolength{\topmargin}{-\headheight}  % 
\addtolength{\topmargin}{-\headsep}     % ヘッダの分だけ本文領域を移動させる
\addtolength{\textheight}{-40truemm}    % 下の余白も20mmに%% 幅の設定
\setlength{\textwidth}{\paperwidth}     % ひとまず紙面を本文領域に
\setlength{\oddsidemargin}{-5.4truemm}  % 左の余白を20mm(=1inch-5.4mm)に
\setlength{\evensidemargin}{-5.4truemm} % 
\addtolength{\textwidth}{-40truemm}     % 右の余白も20mmに
% 図と本文との間
%\abovecaptionskip=-5pt
%\belowcaptionskip=-5pt
%
% 全体の行間調整
% \renewcommand{\baselinestretch}{1.0} 
% 図と表
%\renewcommand{\figurename}{Fig.}
%\renewcommand{\tablename}{Tab.}
%

% \makeatletter 
% \def\section{\@startsection {section}{1}{\z@}{1.5 ex plus 2ex minus -.2ex}{0.5 ex plus .2ex}{\large\bf}}
% \def\subsection{\@startsection{subsection}{2}{\z@}{0.2\Cvs \@plus.5\Cdp \@minus.2\Cdp}{0.1\Cvs \@plus.3\Cdp}{\reset@font\normalsize\bfseries}}
% \makeatother 

\usepackage[dvipdfmx,%
 bookmarks=true,%
 bookmarksnumbered=true,%
 colorlinks=false,%
 setpagesize=false,%
 pdftitle={数式に頼らない直感的理解による材料設計のためのレオロジー⼊⾨},%
 pdfauthor={佐々木裕},%
 pdfsubject={},%
 pdfkeywords={レオロジー; 材料設計; }]{hyperref}
\usepackage{pxjahyper}

\usepackage{plext}

\usepackage{niceframe} 
\usepackage{framed}
\newenvironment{longartdeco}{%
  \def\FrameCommand{\fboxsep=\FrameSep \artdecoframe}%
  \MakeFramed {\FrameRestore}}%
 {\endMakeFramed}
 
\usepackage{siunitx}

\newcommand{\rmd}{\mathrm{d}}

\pagestyle{empty}

\begin{document}

\section*{全体を通してのコメント}
\subsection*{第一章}
\subsubsection*{演習問題1、2について}
ほぼ、理解されているようですので、特に問題はありません。

\subsubsection*{演習問題3について}

ご自身のお仕事を紹介していただき、ありがとうございました。

何を、なんのために、なぜやっているのかということをたまに振り返りながら日々の仕事を進めていけば、問題点も自ずと明らかになってくると思いますので、頑張ってください。


\subsection*{第二章}
\subsubsection*{演習問題1、2について}
ほぼ、理解されているようですので、特に問題はありません。

単位と次元については、きちんとした物理的な議論をするときにとても大事ですので気をつけてください。

\subsubsection*{演習問題3について}

モデル化は、ご自身のイメージを明確にするためにも有効であり、また、他の人と議論するためにも問題点を明確にできるので有効です。

うまく使いこなせるようになってください。

\subsection*{第三章}
\subsubsection*{演習問題1、2について}
ほぼ、理解されているようですので、特に問題はありません。
ただ、少しだけ勘違いが目立ったものが二点ありました。

せん断速度が大きくなると、流れるために必要な応力が大きくなります、すなわち、流れにくくなります。

また、粘度は、測定条件に応じて「常に変化」するわけではなくて、ニュートンの法則が成り立つ範囲であれば一定です。

\subsubsection*{演習問題3について}

固体と液体のそれぞれの性質については、みなさんがきちんと理解されているのですが、シンプルな物理モデルとしての数式的な以下の理解もお願いします。
\begin{itemize}
    \item 固体のモデルである弾性体については、ひずみと応力が比例することと、その比例定数が弾性率であること。
    \item 液体のモデルである流体においては、生じる応力がひずみ速度に比例し、比例定数が流れやすさの指標である粘度であること。
\end{itemize}

\clearpage

\section*{藤森工業株式会社 冨塚様}
\subsection*{第一章}
\subsubsection*{演習問題1、2について}
ほぼ、理解されているようですので、特に問題はありません。

\subsubsection*{演習問題3について}

フィルム処理、及び、その製造に関わられているとのこと。
上手に、数式や図解を使いこなして、知識を広範囲な知見に変えられるようになるといいですね。

頑張ってください。

\subsection*{第二章}
\subsubsection*{演習問題1、2について}
理解されているようですので、特に問題はありません。

\subsubsection*{演習問題3について}

モデル化は、ご自身のイメージを明確にするためにも有効であり、また、他の人と議論するためにも問題点を明確にできるので有効です。

できるだけ、実事象に当てはまるモデルをきちんと構築することが一番大事であり、フックの法則やニュートンの法則も非常にうまく出来たモデル化です。
\subsection*{第三章}
\subsubsection*{演習問題1、2について}
ほぼ、理解されているようですので、特に問題はありません。

ただ、粘度は、測定条件に応じて「常に変化」するわけではなくて、ニュートンの法則が成り立つ範囲であれば一定です。

\subsubsection*{演習問題3について}

固体のモデルである弾性体と液体のモデルである流体について、理解できているようです。

\clearpage
\section*{藤森工業株式会社 横橋様}
\subsection*{第一章}
\subsubsection*{演習問題1、2について}
理解されているようですので、特に問題はありません。

\subsubsection*{演習問題3について}
高分子材料のシート加工の生産検討に関わられているとのこと。
現時点では、まだ、絵の具のようなどろどろした液体と捉えられていると思いますが、高分子材料のレオロジーというものは、かなりややこしいものです。
ニュートン流体として振る舞うのは、かなり限定的な条件だけであり、第三講で少しだけ触れる予定の、粘弾性という粘性と弾性を併せ持って、非ニュートン流体として複雑な流れ方をします。

この初級講座では、残念ながら、高分子材料の物理の細かいところまでは触れることが出来ませんが、まずは基礎的なことの理解から始めてください。
したがって、この講座を終えたとしても、「リスク要因の洗い出し」に直接役立つことはないかもしれません。
でも、ご自分の頭の中で、できるだけ単純化したモデルとして理解しようとするアプローチ自体は、かならず役に立つと思います。
まずは、そのあたりから始めてください。

自由欄に記載されている他の勉強ですが、少しづつでも、高分子の勉強をすすめることが大事かと思います。
しかしながら、初心者の方に役立つような、高分子に関する簡便な教科書のいいものがあまりないんです。
教科書紹介の記事のリンクを張ります。\\
\url{https://www.jstage.jst.go.jp/article/kobunshi1952/54/4/54_4_250/_pdf}

\subsection*{第二章}
\subsubsection*{演習問題1、2について}
理解されているようですので、特に問題はありません。

\subsubsection*{演習問題3について}

モデル化は、ご自身のイメージを明確にするためにも有効であり、また、他の人と議論するためにも問題点を明確にできるので有効です。
ただ、一回構築すればそれで終わりということではなく、常に改善するという不断の努力を続けないと、同じ間違いを犯してしまうかもしれません。
頑張ってください。

\subsection*{第三章}
\subsubsection*{演習問題1、2について}
ほぼ、理解されているようですが、以下の点には気をつけてください。

せん断速度が大きくなると、流れるために必要な応力が大きくなり、流れにくくなります。
また、粘度は、測定条件に応じて「常に変化」するわけではなくて、ニュートンの法則が成り立つ範囲であれば一定です。

応力については、入れ替えても一応文章は成り立ちますが、物理量としては力の大きさとしたほうが適切かと。

\subsubsection*{演習問題3について}

固体のモデルである弾性体と液体のモデルである流体について、理解できているようです。

\clearpage

\section*{藤森工業株式会社 池田様}
\subsection*{第一章}
\subsubsection*{演習問題1、2について}
ほぼ、理解されているようですので、特に問題はありません。

\subsubsection*{演習問題3について}

製品の特性評価や物性測定に関わられているのですね。

塗料のレオロジーは、非常に複雑です。
ニュートン流体として振る舞うのは、かなり限定的な条件だけであり、第三講で少しだけ触れる予定の非ニュートン流体として複雑な流れ方をしますし、粘弾性という粘性と弾性を併せ持った声質を持っています。
この初級講座では、残念ながら、高分子材料の物理の細かいところまでは触れることが出来ませんが、まずは基礎的なことの理解から始めてください。

ご自分の頭の中で、できるだけ単純化したモデルとして理解しようとするアプローチ自体は、かならず役に立つと思います。
まずは、そのあたりから始めてください。

\subsection*{第二章}
\subsubsection*{演習問題1、2について}
理解されているようですので、特に問題はありません。

\subsubsection*{演習問題3について}

モデル化は、ご自身のイメージを明確にするためにも有効であり、また、他の人と議論するためにも問題点を明確にできるので有効です。

ただ、一回構築すればそれで終わりということではなく、常に改善するという不断の努力を続けないと、同じ間違いを犯してしまうかもしれません。
頑張ってください。

\subsection*{第三章}
\subsubsection*{演習問題1、2について}
ほぼ、理解されているようですが、以下の点には気をつけてください。

せん断速度が大きくなると、流れるために必要な応力が大きくなります、すなわち、流れにくくなります。
また、粘度は、測定条件に応じて「常に変化」するわけではなくて、ニュートンの法則が成り立つ範囲であれば一定です。

\subsubsection*{演習問題3について}
固体のモデルである弾性体については、理解できているようです。

液体のモデルである流体においては、ニュートンの法則が成り立つような単純な流れ方においては、生じる応力がひずみ速度に比例します。
第三講で少しだけ触れる予定の非ニュートン流体では、たしかに、ひずみ速度に応じて流れ方の指標である粘度も変化して、複雑な流れ方をしますが、まずは、単純なモデルをきちんと理解してください。

\clearpage

\section*{テルモ株式会社 杉浦様}
\subsection*{第一章}
\subsubsection*{演習問題1、2について}
理解されているようですので、問題はありません。

\subsubsection*{演習問題3について}

血糖測定器の開発に関わられているのですね。

ぜひとも、広く学んで柔軟な考え方を身につけて、多様な知見を蓄えていってください。

テキストの語り口に好感をもっていただき、ありがとうございます。
今後とも宜しくお願いいたします。

\subsection*{第二章}
\subsubsection*{演習問題1、2について}
ほぼ、理解されているようですが、以下の点には気をつけてください。

単位とは、「\textbf{同種の物理量}の大きさを表すため」に取り決めによって定義されたものですから、単純に量でもなく、特定の会社間でもありません。

\subsubsection*{演習問題3について}

モデル化は、ご自身のイメージを明確にするためにも有効であり、また、他の人と議論するためにも問題点を明確にできるので有効です。
本質に迫れるように、頑張ってください。

\subsection*{第三章}
\subsubsection*{演習問題1、2について}
ほぼ、理解されているようですが、以下の点には気をつけてください。

せん断速度が大きくなると、流れるために必要な応力が大きくなります、すなわち、流れにくくなります。
また、粘度は、測定条件に応じて「常に変化」するわけではなくて、ニュートンの法則が成り立つ範囲であれば一定です。

\subsubsection*{演習問題3について}
力学モデルについて、もう少し理解を深めたほうがいいかもしれません。

固体のモデルである弾性体については、ひずみと応力が比例することと、その比例定数が弾性率であること。
液体のモデルである流体においては、生じる応力がひずみ速度に比例し、比例定数が流れやすさの指標である粘度であること。

\clearpage

\section*{テルモ株式会社 洞口様}
\subsection*{第一章}
\subsubsection*{演習問題1、2について}
理解されているようですので、問題はありません。

\subsubsection*{演習問題3について}

糖尿病関連の機器の開発に関わられているのですね。

ぜひとも、広く学んで、物性を定量的に捉えられるようになって、安定生産につながるようなパラメタをうまく捕まえてください。

\subsection*{第二章}
\subsubsection*{演習問題1、2について}
理解されているようですので、特に問題はありません。

\subsubsection*{演習問題3について}

モデル化は、ご自身のイメージを明確にするためにも有効であり、また、他の人と議論するためにも問題点を明確にできるので有効です。
適切なモデルかどうかをきちんと見極められるように、頑張ってください。

\subsection*{第三章}
\subsubsection*{演習問題1、2について}
ほぼ、理解されているようですが、以下の点には気をつけてください。

粘度は、測定条件に応じて「常に変化」するわけではなくて、ニュートンの法則が成り立つ範囲であれば一定です。

\subsubsection*{演習問題3について}
それぞれの性質はきちんと理解されているのですが、シンプルな物理モデルとして以下の理解も必要かと思います。

固体のモデルである弾性体については、ひずみと応力が比例することと、その比例定数が弾性率であること。
液体のモデルである流体においては、生じる応力がひずみ速度に比例し、比例定数が流れやすさの指標である粘度であること。

\clearpage

\section*{テルモ株式会社 茶井様}
\subsection*{第一章}
\subsubsection*{演習問題1、2について}
理解されているようですので、問題はありません。

\subsubsection*{演習問題3について}

糖尿病関連のデバイスの開発に関わられているのですね。

ぜひとも、定性的な違いを定量的に捉えて共通認識とできるように、安定生産につながるようなパラメタをうまく捕まえてください。

\subsection*{第二章}
\subsubsection*{演習問題1、2について}
ほぼ、理解されているようですので、特に問題はありません。

\subsubsection*{演習問題3について}

モデル化は、ご自身のイメージを明確にするためにも有効であり、また、他の人と議論するためにも問題点を明確にできるので有効です。
実事象を完全にモデル化することは不可能ですので、適切なモデルかどうかをきちんと見極められるように、頑張ってください。

\subsection*{第三章}
\subsubsection*{演習問題1、2について}
ほぼ、理解されているようですが、以下の点には気をつけてください。

粘度は、測定条件に応じて「常に変化」するわけではなくて、ニュートンの法則が成り立つ範囲であれば一定です。

\subsubsection*{演習問題3について}
それぞれの性質はきちんと理解されているのですが、シンプルな物理モデルとして以下の理解も必要かと思います。

固体のモデルである弾性体については、ひずみと応力が比例することと、その比例定数が弾性率であること。
液体のモデルである流体においては、生じる応力がひずみ速度に比例し、比例定数が流れやすさの指標である粘度であること。

\clearpage

\section*{テルモ株式会社 津田様}
\subsection*{第一章}
\subsubsection*{演習問題1、2について}
理解されているようですので、問題はありません。

\subsubsection*{演習問題3について}

自社製の測定器の測定結果にコミットするようなお仕事をされているのですね。

ぜひとも、ブラックボックスになりがちな内部での減少を可視化できるようにすることで、根拠が明示できるような議論を行って、信頼関係を構築してください。

\subsection*{第二章}
\subsubsection*{演習問題1、2について}
ほぼ、理解されているようですので、特に問題はありません。

\subsubsection*{演習問題3について}

モデル化は、ご自身のイメージを明確にするためにも有効であり、また、他の人と議論するためにも問題点を明確にできるので有効です。

実事象を完全にモデル化することは不可能ですので、原理が理解できるような、単純化して可視化できるような適切なモデルを構築できるように、頑張ってください。

\subsection*{第三章}
\subsubsection*{演習問題1、2について}
ほぼ、理解されているようですが、以下の点には気をつけてください。

せん断速度が大きくなると、流れるために必要な応力が大きくなります、すなわち、流れにくくなります。
また、粘度は、測定条件に応じて「常に変化」するわけではなくて、ニュートンの法則が成り立つ範囲であれば一定です。

\subsubsection*{演習問題3について}
それぞれの性質はきちんと理解されているのですが、シンプルな物理モデルとして以下の理解も必要かと思います。

固体のモデルである弾性体については、ひずみと応力が比例することと、その比例定数が弾性率であること。
液体のモデルである流体においては、生じる応力がひずみ速度に比例し、比例定数が流れやすさの指標である粘度であること。

もう一度だけ、本文を確認してください。

応力の単位は同一です。
比例定数の単位が異なります。
\clearpage
\end{document}