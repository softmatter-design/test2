\documentclass[uplatex,dvipdfmx,a4paper,11pt]{jsreport}

\usepackage{docmute}


% 数式
\usepackage{amsmath,amsthm,amssymb}
\usepackage{bm}
% 画像
\usepackage{graphicx}

\usepackage{multirow}
\usepackage{wrapfig}
\usepackage{ascmac}
\usepackage{xcolor}


\usepackage{makeidx}
\makeindex

\graphicspath{{../../Figures//}{../../Figures/Rheology/}}

\usepackage{qrcode}
\setlength\lineskiplimit{0pt}
\setlength\normallineskiplimit{0pt}

\usepackage{qexam}

\usepackage{titlesec}
\titleformat*{\section}{\Large\bfseries}
\titleformat*{\subsection}{\large\bfseries}
\titleformat*{\subsubsection}{\normalsize\bfseries}
\titleformat*{\paragraph}{\normalsize\bfseries}

% ページ設定
% \pagestyle{empty}
% 高さの設定
\setlength{\textheight}{\paperheight}   % ひとまず紙面を本文領域に
\setlength{\topmargin}{-5.4truemm}      % 上の余白を20mm(=1inch-5.4mm)に
\addtolength{\topmargin}{-\headheight}  % 
\addtolength{\topmargin}{-\headsep}     % ヘッダの分だけ本文領域を移動させる
\addtolength{\textheight}{-40truemm}    % 下の余白も20mmに%% 幅の設定
\setlength{\textwidth}{\paperwidth}     % ひとまず紙面を本文領域に
\setlength{\oddsidemargin}{-5.4truemm}  % 左の余白を20mm(=1inch-5.4mm)に
\setlength{\evensidemargin}{-5.4truemm} % 
\addtolength{\textwidth}{-40truemm}     % 右の余白も20mmに
% 図と本文との間
%\abovecaptionskip=-5pt
%\belowcaptionskip=-5pt
%
% 全体の行間調整
% \renewcommand{\baselinestretch}{1.0} 
% 図と表
%\renewcommand{\figurename}{Fig.}
%\renewcommand{\tablename}{Tab.}
%

% \makeatletter 
% \def\section{\@startsection {section}{1}{\z@}{1.5 ex plus 2ex minus -.2ex}{0.5 ex plus .2ex}{\large\bf}}
% \def\subsection{\@startsection{subsection}{2}{\z@}{0.2\Cvs \@plus.5\Cdp \@minus.2\Cdp}{0.1\Cvs \@plus.3\Cdp}{\reset@font\normalsize\bfseries}}
% \makeatother 

\usepackage[dvipdfmx,%
 bookmarks=true,%
 bookmarksnumbered=true,%
 colorlinks=false,%
 setpagesize=false,%
 pdftitle={数式に頼らない直感的理解による材料設計のためのレオロジー⼊⾨},%
 pdfauthor={佐々木裕},%
 pdfsubject={},%
 pdfkeywords={レオロジー; 材料設計; }]{hyperref}
\usepackage{pxjahyper}

\usepackage{plext}

\usepackage{niceframe} 
\usepackage{framed}
\newenvironment{longartdeco}{%
  \def\FrameCommand{\fboxsep=\FrameSep \artdecoframe}%
  \MakeFramed {\FrameRestore}}%
 {\endMakeFramed}
 
\usepackage{siunitx}

\newcommand{\rmd}{\mathrm{d}}

\pagestyle{empty}

\begin{document}

\section*{全体を通してのコメント}
\subsection*{第一章}
\subsubsection*{演習問題1、2について}
ほぼ、理解されているようですので、特に問題はありません。

ポテンシャルは、力が「保存力」であるときだけ「位置のみの関数として与えられる状態量」となることに注意してください。

\subsection*{第二章}
\subsubsection*{演習問題1、2について}
ほぼ、理解されているようですので、特に問題はありません。



\subsubsection*{演習問題3について}

皆さん、ご自身の中で、きちんとイメージを持たれているようです。
全く問題ありません。



\clearpage

\section*{藤森工業株式会社 冨塚様}
\subsection*{第一章}
\subsubsection*{演習問題1、2について}
ほぼ、理解されているようですので、特に問題はありません。

他の定義を用いることも出来ますが、この講義においては、「緩和時間」を初期の 1/e になる時間と定義しています。

また、ポテンシャルは、力が「保存力」であるときだけ「位置のみの関数として与えられる状態量」となることに注意してください。


\subsection*{第二章}
\subsubsection*{演習問題1、2について}
理解されているようですので、特に問題はありません。

\subsubsection*{演習問題3について}

ご自身の中で、きちんとイメージを持たれているようです。
全く問題ありません。

\clearpage
\section*{藤森工業株式会社 横橋様}
\subsection*{第一章}
\subsubsection*{演習問題1、2について}
ほぼ理解されているようですので、特に問題はありません。

ポテンシャルは、力が「保存力」であるときだけ「位置のみの関数として与えられる状態量」となることに注意してください。


\subsection*{第二章}
\subsubsection*{演習問題1、2について}
理解されているようですので、特に問題はありません。

\subsubsection*{演習問題3について}

大枠のイメージは理解されているようです。

マクロな変形とミクロに生じていることとをもう少し分けてイメージしたほうがいいかもしれません。

\clearpage

\section*{藤森工業株式会社 池田様}
\subsection*{第一章}
\subsubsection*{演習問題1、2について}
ほぼ、理解されているようですので、特に問題はありません。


\subsection*{第二章}
\subsubsection*{演習問題1、2について}
ほぼ、理解されているようですので、特に問題はありません。

液体が流れるときには、「内部の粒子が瞬間ごとの居心地のいい状態に移動している」ことに注意してください。

\subsubsection*{演習問題3について}

ご自身の中で、きちんとイメージを持たれているようです。
全く問題ありません。

\clearpage

\section*{テルモ株式会社 杉浦様}
\subsection*{第一章}
\subsubsection*{演習問題1、2について}
ほぼ理解されているようですので、特に問題はありません。

ポテンシャルは、力が「保存力」であるときだけ「位置のみの関数として与えられる状態量」となることに注意してください。

\subsection*{第二章}
\subsubsection*{演習問題1、2について}
理解されているようですので、問題はありません。

\subsubsection*{演習問題3について}

ご自身の中で、きちんとイメージを持たれているようです。
全く問題ありません。

自発的な表現と他動的なものを両方イメージすることは大事だと思います。

\clearpage

\section*{テルモ株式会社 洞口様}
\subsection*{第一章}
\subsubsection*{演習問題1、2について}
ほぼ、理解されているようですので、特に問題はありません。

ポテンシャルは、力が「保存力」であるときだけ「位置のみの関数として与えられる状態量」となることに注意してください。

\subsection*{第二章}
\subsubsection*{演習問題1、2について}
理解されているようですので、問題はありません。

\subsubsection*{演習問題3について}

ご自身の中で、きちんとイメージを持たれているようです。
全く問題ありません。

\clearpage

\section*{テルモ株式会社 茶井様}
\subsection*{第一章}
\subsubsection*{演習問題1、2について}
理解されているようですので、問題はありません。

\subsection*{第二章}
\subsubsection*{演習問題1、2について}
理解されているようですので、問題はありません。

\subsubsection*{演習問題3について}

ご自身の中で、きちんとイメージを持たれているようです。
全く問題ありません。

\clearpage

\section*{テルモ株式会社 津田様}
\subsection*{第一章}
\subsubsection*{演習問題1、2について}
ほぼ、理解されているようですので、特に問題はありません。

ポテンシャルは、力が「保存力」であるときだけ「位置のみの関数として与えられる状態量」となることに注意してください。


\subsection*{第二章}
\subsubsection*{演習問題1、2について}
ほぼ、理解されているようですので、特に問題はありません。

固体と比較して、「熱の影響が相対的に大きくてそれぞれの粒子が一箇所に留まらない状態が液体である」ことに注意してください。

% また、固体と液体の相転移に於いて、熱的状態が変化したときに「運動状態が変化して、パッキングも変わる」と考えたほうが自然かなと思います。

\subsubsection*{演習問題3について}

一行目の表記は、若干適切ではないかもしれません。
液体において、個々の粒子は常に居心地の良い場所を求めて移動していますが、これは、「流れていなくても生じている」ことです。

二行目のイメージは、素晴らしい。
このように活性化されるということを意識してください。

で、流れるということに着目したときには、上記の移動が等方的(どの方向も平等)ではなく、異方的(特定の方向にエコヒイキ)になるということと捉えればいいと思います。
その原因として、巨視的な変形に伴って居心地の良い方向が出来ているということと考えてください。


\clearpage
\end{document}