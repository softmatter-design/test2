\documentclass[uplatex,dvipdfmx,a4paper,11pt]{jsarticle}

\usepackage{docmute}


% 数式
\usepackage{amsmath,amsthm,amssymb}
\usepackage{bm}
% 画像
\usepackage{graphicx}

\usepackage{multirow}
\usepackage{wrapfig}
\usepackage{ascmac}
\usepackage{xcolor}


\usepackage{makeidx}
\makeindex

\graphicspath{{../../Figures//}{../../Figures/Rheology/}}

\usepackage{qrcode}
\setlength\lineskiplimit{0pt}
\setlength\normallineskiplimit{0pt}

\usepackage{qexam}

\usepackage{titlesec}
\titleformat*{\section}{\Large\bfseries}
\titleformat*{\subsection}{\large\bfseries}
\titleformat*{\subsubsection}{\normalsize\bfseries}
\titleformat*{\paragraph}{\normalsize\bfseries}

% ページ設定
% \pagestyle{empty}
% 高さの設定
\setlength{\textheight}{\paperheight}   % ひとまず紙面を本文領域に
\setlength{\topmargin}{-5.4truemm}      % 上の余白を20mm(=1inch-5.4mm)に
\addtolength{\topmargin}{-\headheight}  % 
\addtolength{\topmargin}{-\headsep}     % ヘッダの分だけ本文領域を移動させる
\addtolength{\textheight}{-40truemm}    % 下の余白も20mmに%% 幅の設定
\setlength{\textwidth}{\paperwidth}     % ひとまず紙面を本文領域に
\setlength{\oddsidemargin}{-5.4truemm}  % 左の余白を20mm(=1inch-5.4mm)に
\setlength{\evensidemargin}{-5.4truemm} % 
\addtolength{\textwidth}{-40truemm}     % 右の余白も20mmに
% 図と本文との間
%\abovecaptionskip=-5pt
%\belowcaptionskip=-5pt
%
% 全体の行間調整
% \renewcommand{\baselinestretch}{1.0} 
% 図と表
%\renewcommand{\figurename}{Fig.}
%\renewcommand{\tablename}{Tab.}
%

% \makeatletter 
% \def\section{\@startsection {section}{1}{\z@}{1.5 ex plus 2ex minus -.2ex}{0.5 ex plus .2ex}{\large\bf}}
% \def\subsection{\@startsection{subsection}{2}{\z@}{0.2\Cvs \@plus.5\Cdp \@minus.2\Cdp}{0.1\Cvs \@plus.3\Cdp}{\reset@font\normalsize\bfseries}}
% \makeatother 

\usepackage[dvipdfmx,%
 bookmarks=true,%
 bookmarksnumbered=true,%
 colorlinks=false,%
 setpagesize=false,%
 pdftitle={数式に頼らない直感的理解による材料設計のためのレオロジー⼊⾨},%
 pdfauthor={佐々木裕},%
 pdfsubject={},%
 pdfkeywords={レオロジー; 材料設計; }]{hyperref}
\usepackage{pxjahyper}

\usepackage{plext}

\usepackage{niceframe} 
\usepackage{framed}
\newenvironment{longartdeco}{%
  \def\FrameCommand{\fboxsep=\FrameSep \artdecoframe}%
  \MakeFramed {\FrameRestore}}%
 {\endMakeFramed}
 
\usepackage{siunitx}

\newcommand{\rmd}{\mathrm{d}}

\usepackage[inline]{showlabels}

\begin{document}

\question{演習問題 1}
内容を振り返るために、以下に示した文章例の中から適切な記述のものを複数選んでください。
	\begin{qlist}
		\qitem 指数関数と対数関数についての、正しい言葉はどれでしょうか?
		\begin{qlist2}
			\qitem 指数関数とは、「底」と呼ばれる正の数の右肩に「指数」と呼ばれる数を載せた数式表現です。
			\qitem 対数関数とは、指数関数の逆関数になっています。
			\qitem 対数関数は、指数関数に反比例します。
			\qitem 
			\qitem 
		\end{qlist2}
    \vspace{3mm}
	\qitem 
		\begin{qlist2}
			\qitem 
			\qitem 
			\qitem 
			\qitem 
			\qitem 
		\end{qlist2}
    \vspace{3mm}
	\qitem 
		\begin{qlist2}
			\qitem 
			\qitem 
			\qitem 
			\qitem 
			\qitem 
		\end{qlist2}
    \vspace{3mm}
	\qitem 
	\begin{qlist2}
		\qitem 
		\qitem 
		\qitem 
		\qitem 
		\qitem 
	\end{qlist2}
  \vspace{3mm}
	\qitem 
	\begin{qlist2}
		\qitem 
		\qitem 
		\qitem 
		\qitem 
		\qitem 
	\end{qlist2}
\end{qlist}

\question{演習問題 2}
内容を振り返るために、テキストで用いた言葉を使って簡単な穴埋めを行ってください。

\begin{qparts}
\qpart 「弾性体の力学的な刺激と応答」について、以下の\qbox{(a)}から\qbox{(i)}までのカッコを埋めてください。
		\begin{qlist}
			\qitem 変形は、物質を一つの軸に沿って引き伸ばす「\qbox{}」とトランプのカードを横にずらしたような「\qbox{}」の二つに単純化できます。
			\qitem 伸張変形では、\qbox{}を\qbox{}で除したものがひずみとなります。
			\qitem 応力とは、物質の内部に生じている\qbox{}を表す物理量であり、その表す意味は単位面積あたりの\qbox{}ということになります。
			\qitem 応力と力の関係は以下のように書けます。
			\begin{align*}
				\qbox{} = \dfrac{\qbox{}}{\qbox{}}
			\end{align*}
			% \qitem 一様な太さの棒を引っ張ったとき、棒の長手方向にはどの位置で切断したとしても、\qbox{}が働いていることに注意してください。

      \begin{itembox}[l]{選択肢}
        \begin{center}
          \begin{tabular}{lllll}
            1. 変形量	&2. せん断変形	&3. 力の大きさ	&4. 面積	&5. 応力\\
            6. 初期長さ	&7. 力		&8. 伸張変形				&9. 内部の力
          \end{tabular}
        \end{center}
      \end{itembox}

    \end{qlist}

\qpart 弾性体と液体の力学応答について、以下の\qbox{(j)}から\qbox{(p)}までのカッコを埋めてください。
		\begin{qlist}
			\qitem 弾性体を表すモデルはフックの法則であり、以下のように書けます。
				\begin{center}
					\begin{minipage}{0.45\textwidth}
						\begin{itemize}
							\item \qbox{} $\varepsilon$ と比例して、
							\item \qbox{} $\sigma$ が生じ、
							\item 比例定数が\qbox{} $E$
						\end{itemize}
						\begin{align*}
							\sigma = E \varepsilon
						\end{align*}
					\end{minipage}
					\begin{minipage}{0.35\textwidth}
						\begin{center}
						\includegraphics[width=.8\textwidth]{hook_law.png}
						\end{center}
					\end{minipage}
				\end{center}
			\qitem 液体の評価は主としてそのひずみ速度を維持しやすい「\qbox{}」により行われる場合が多くなります。
			\qitem ニュートンの法則の比例関係を式で表せば以下のようになります。
			\begin{align*}
				\text{\qbox{}} &= \text{\qbox{}} \times \text{\qbox{}} \notag \\
				\tau &= \eta \dot{\gamma}
      \end{align*}
      
      \begin{itembox}[l]{選択肢}
        \begin{center}
          \begin{tabular}{llll}
            1. 伸張弾性率	&2. ひずみ速度	&3. 伸張ひずみ	&4. せん断応力\\
            5. 伸張応力 	&6. せん断変形	&7. 粘度		
          \end{tabular}
        \end{center}
      \end{itembox}
  \end{qlist}

\end{qparts}

\question{演習問題 3}
数行程度の簡単な記述で構いませんので、以下の自由記述問題を考えてみてください。
\begin{qlist}
\qitem この章では、レオロジーのはじめの一歩として、その力学モデルについて簡単な説明を行ってきました。
弾性体と液体の力学モデルの特徴についてそれぞれ簡単にまとめて、その大きな違いについて書いてみてください。
\end{qlist}

\clearpage

\end{document}