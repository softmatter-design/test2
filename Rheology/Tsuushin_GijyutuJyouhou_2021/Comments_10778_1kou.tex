\documentclass[uplatex,dvipdfmx,a4paper,11pt]{jsreport}

\usepackage{docmute}


% 数式
\usepackage{amsmath,amsthm,amssymb}
\usepackage{bm}
% 画像
\usepackage{graphicx}

\usepackage{multirow}
\usepackage{wrapfig}
\usepackage{ascmac}
\usepackage{xcolor}


\usepackage{makeidx}
\makeindex

\graphicspath{{../../_Figures//}{../../_Figures/Rheology/}}

\usepackage{qrcode}
\setlength\lineskiplimit{0pt}
\setlength\normallineskiplimit{0pt}

\usepackage{qexam}

\usepackage{titlesec}
\titleformat*{\section}{\Large\bfseries}
\titleformat*{\subsection}{\large\bfseries}
\titleformat*{\subsubsection}{\normalsize\bfseries}
\titleformat*{\paragraph}{\normalsize\bfseries}

% ページ設定
% \pagestyle{empty}
% 高さの設定
\setlength{\textheight}{\paperheight}   % ひとまず紙面を本文領域に
\setlength{\topmargin}{-5.4truemm}      % 上の余白を20mm(=1inch-5.4mm)に
\addtolength{\topmargin}{-\headheight}  % 
\addtolength{\topmargin}{-\headsep}     % ヘッダの分だけ本文領域を移動させる
\addtolength{\textheight}{-40truemm}    % 下の余白も20mmに%% 幅の設定
\setlength{\textwidth}{\paperwidth}     % ひとまず紙面を本文領域に
\setlength{\oddsidemargin}{-5.4truemm}  % 左の余白を20mm(=1inch-5.4mm)に
\setlength{\evensidemargin}{-5.4truemm} % 
\addtolength{\textwidth}{-40truemm}     % 右の余白も20mmに
% 図と本文との間
%\abovecaptionskip=-5pt
%\belowcaptionskip=-5pt
%
% 全体の行間調整
% \renewcommand{\baselinestretch}{1.0} 
% 図と表
%\renewcommand{\figurename}{Fig.}
%\renewcommand{\tablename}{Tab.}
%

% \makeatletter 
% \def\section{\@startsection {section}{1}{\z@}{1.5 ex plus 2ex minus -.2ex}{0.5 ex plus .2ex}{\large\bf}}
% \def\subsection{\@startsection{subsection}{2}{\z@}{0.2\Cvs \@plus.5\Cdp \@minus.2\Cdp}{0.1\Cvs \@plus.3\Cdp}{\reset@font\normalsize\bfseries}}
% \makeatother 

\usepackage[dvipdfmx,%
 bookmarks=true,%
 bookmarksnumbered=true,%
 colorlinks=false,%
 setpagesize=false,%
 pdftitle={数式に頼らない直感的理解による材料設計のためのレオロジー⼊⾨},%
 pdfauthor={佐々木裕},%
 pdfsubject={},%
 pdfkeywords={レオロジー; 材料設計; }]{hyperref}
\usepackage{pxjahyper}

\usepackage{plext}

\usepackage{niceframe} 
\usepackage{framed}
\newenvironment{longartdeco}{%
  \def\FrameCommand{\fboxsep=\FrameSep \artdecoframe}%
  \MakeFramed {\FrameRestore}}%
 {\endMakeFramed}
 
\usepackage{siunitx}

\newcommand{\rmd}{\mathrm{d}}

\pagestyle{empty}

\begin{document}

\section*{全体を通してのコメント}
\subsection*{第一章}
\subsubsection*{演習問題1、2について}
ほぼ、理解されているようですので、特に問題はありません。

\subsubsection*{演習問題3について}

ご自身のお仕事を紹介していただき、ありがとうございました。

何を、なんのために、なぜやっているのかということをたまに振り返りながら日々の仕事を進めていけば、
問題点も自ずと明らかになってくると思いますので、頑張ってください。


\subsection*{第二章}
\subsubsection*{演習問題1、2について}
ほぼ、理解されているようですので、特に問題はありません。

単位と次元については、きちんとした物理的な議論をするときにとても大事ですので気をつけてください。

\subsubsection*{演習問題3について}

モデル化は、ご自身のイメージを明確にするためにも有効であり、また、他の人と議論するためにも問題点を明確にできるので有効です。

うまく使いこなせるようになってください。

\subsection*{第三章}
\subsubsection*{演習問題1、2について}
ほぼ、理解されているようですので、特に問題はありません。
ただ、少しだけ勘違いが目立ったものが二点ありました。

せん断速度が大きくなると、流れるために必要な応力が大きくなります、すなわち、流れにくくなります。

また、粘度は、測定条件に応じて「常に変化」するわけではなくて、ニュートンの法則が成り立つ範囲であれば一定です。

\subsubsection*{演習問題3について}

固体と液体のそれぞれの性質については、みなさんがきちんと理解されているのですが、シンプルな物理モデルとしての数式的な以下の理解も
お願いします。
\begin{itemize}
    \item 固体のモデルである弾性体については、ひずみと応力が比例することと、その比例定数が弾性率であること。
    \item 液体のモデルである流体においては、生じる応力がひずみ速度に比例し、比例定数が流れやすさの指標である粘度であること。
\end{itemize}

\clearpage

\section*{江森 様}
\subsection*{第一章}
\subsubsection*{演習問題1、2について}
ほぼ、理解されているようですので、特に問題はありません。

\subsubsection*{演習問題3について}

製品及び素材関連の試験、検査、分析に関わられているとのこと。
ご自身のお仕事を進化させていくのに、レオロジーの知識がうまく役立つといいですね。

今回の通信教育では、残念ながら動的な測定についてまでは到達できませんので、周波数については触れることはできません。
時間温度換算速も明示的には議論するところまでもお話は進みませんが、その手前の事項として、液体が流れるという現象に時間の因子が重要であること。
さらに、それは、粒子の運動と強い関連があるので、温度とも関連が強いことについては説明を進めていきます。
そのような基礎的な知見を身に着けていただくことを、今回の通信教育では目指しています。

頑張ってください。

\subsection*{第二章}
\subsubsection*{演習問題1、2について}
理解されているようですので、特に問題はありません。

\subsubsection*{演習問題3について}

モデル化は、ご自身のイメージを明確にするためにも有効であり、また、他の人と議論するためにも問題点を明確にできるので有効です。

できるだけ、実事象に当てはまるモデルをきちんと構築することが一番大事であり、フックの法則やニュートンの法則も非常にうまく出来たモデル化です。

ご自分の言葉で、自分用のモデル化ができるように頑張ってください。

\subsection*{第三章}
\subsubsection*{演習問題1、2について}
ほぼ、理解されているようですので、特に問題はありません。

ただ、粘度は、測定条件に応じて「常に変化」するわけではなくて、ニュートンの法則が成り立つ範囲であれば一定です。

\subsubsection*{演習問題3について}

固体のモデルである弾性体と液体のモデルである流体について、若干の言葉の齟齬があるのかもしれません。

「液体が変形を維持しやすい」と考えるのではなくて、一旦変形してしまうともとに戻るすべはないのです。
で、その流動するという現象の中に、時間の因子が入ってきます。

まあ、流れるという特性があるので、測定はせん断ひずみで行うのが簡便となっているのです。

\clearpage
\section*{平井 様}
\subsection*{第一章}
\subsubsection*{演習問題1、2について}
理解されているようですので、特に問題はありません。

\subsubsection*{演習問題3について}
ゴムやプラスチックの耐久性や各種の振動特性等を検討されているのですね。
ご自身の理解を深めて、依頼者の方への説明等に生かされたいとの事、了解しました。

この初級講座では、残念ながら、高分子材料の物理の細かいところまでは触れることが出来ませんが、まずは基礎的なことの理解から始めてください。
したがって、この講座を終えたとしても、ゴム力学や動的な疲労試験の理解に対して、直接役立つことはないかもしれません。
でも、ご自分の頭の中で、できるだけ単純化したモデルとして理解しようとするアプローチ自体は、かならず役に立つと思います。
まずは、そのあたりから始めてください。

% しかしながら、初心者の方に役立つような、高分子に関する簡便な教科書のいいものがあまりないんです。
% 教科書紹介の記事のリンクを張ります。\\
% \url{https://www.jstage.jst.go.jp/article/kobunshi1952/54/4/54_4_250/_pdf}

\subsection*{第二章}
\subsubsection*{演習問題1、2について}
理解されているようですので、特に問題はありません。

\subsubsection*{演習問題3について}

モデル化は、ご自身のイメージを明確にするためにも有効であり、また、他の人と議論するためにも問題点を明確にできるので有効です。
レオロジーの教科書もすでに学ばれているようですが、前述のように、この講座はその前の段階を深く理解することを目指していますので、
少しご希望には添えないかもしれません。
申し訳ありません。

\subsection*{第三章}
\subsubsection*{演習問題1、2について}
ほぼ、理解されているようので、特に問題はありません。

\subsubsection*{演習問題3について}

レオロジーのいちばん大事な点である緩和現象を理解するためには、液体の応答の過程である流動現象を、マクロとミクロの両面から理解することが重要であると私は考えています。

簡単すぎるようなモデルに見えるかもしれませんが、ダッシュポットモデルが示す挙動について、よく理解すれば、あなたが一番興味のある耐久性に
強く関与する緩和現象の理解に必ず役立つと思います。

\clearpage

\end{document}