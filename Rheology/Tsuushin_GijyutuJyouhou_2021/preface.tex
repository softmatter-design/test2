\chapter*{まえがき}

\section*{レオロジーとは}
我々の身の回りの材料の大半は流れるという性質を持っています。
当然、それぞれの材料ごとに流れ易さは異なりますが、非常に長い時間をかければ不動に見える岩や大地も流れていきます。
その流れるものを測る学問がレオロジーです。

レオロジーは「お触りの科学」とも言われています。
人間の五感(特に触覚)は極めて優秀であり、手触りで物質の特徴の違いを直感的に区別することができます。
しかしながら、この直感的な区別を材料や商品の開発へと結びつけることは困難です。
それは、直感的な区別はあくまでも定性的なものであり、定量性に欠ける場合がほとんどであるためです。

\section*{目指すもの}
レオロジーの本質をきちんと理解することで、各種の材料の違いを明確に区別する方法がイメージでき、その違いを利用した材料の設計のポイントもわかってきます。

直感的に感じる違いをきちんとした理解へと結びつけるために、本セミナーでは、「箇条書き」や「図解」を多用します。
そうすることで、単に抽象的な概念としてだけではなく、ブレイクダウンしたイメージとして直感的に捉えていきます。
また、数式だけに頼ることなく、数式が表したいことを理解して、イメージと数式をつなげていきます。

本セミナーは、レオロジーを実践的に使いこなすためのベースとなる基本的な事項を実感として理解し、材料の持つ「流動と弾性」という二面性をイメージとして持てるようになることを目指します。

\section*{簡単な自己紹介}

筆者は、もともと合成化学をベースとした高分子化学を専攻し、1986 年に化学系材料のメーカーである東亞合成株式会社に入社しました。
入社当初の仕事としては、新規構造を有する材料の合成を主として各種の光硬化型材料の研究開発を行ってきました。
その自身が合成した材料の特性評価を行う過程において、レオロジーを始めとした各種の材料評価技術の習得へとつなげてきています。

これらの経験において、「新規なものを作り出す技術としての化学の有用性」を何度も再認識してきました。
それと同時に、物理、数学、統計等の中にある「事象を客観視しながら普遍性を大事にするものの考え方」の重要性を痛感する場面も幾度となく経験してきました。
そして、それらの場面において最も役立ったのは、レオロジーから学んだマクロな応答をミクロな化学構造へと繋げられる想像力でした。

このような感覚を皆さんにお伝えできればと望んでおります。

\begin{screen}
	\begin{itemize}
		\item 略歴
        \begin{itemize}
            \item 北海道大学で合成化学系の高分子化学を専攻
			\item 卒業後、東亞合成株式会社に入社し、現在に至る
		\end{itemize}
		\item  研究・開発歴
		\begin{itemize}
			\item 合成をベースとした各種の光硬化型材料の研究開発に従事。
			\item 機能性材料の特性評価を通して、レオロジー等の評価技術の重要性を痛感。
			\item 現在は、シミュレーションやレオロジーを主として研究活動を継続。
		\end{itemize}
		\item モットー
		\begin{itemize}
            \item 「化学をベースに、尤もらしく」
			\item 「物理、数学、統計の考えを利用して」
		\end{itemize}
	\end{itemize}
\end{screen}

\section*{本講座の進め方}

本講座においては、実際の研究開発に役に立つレオロジー関連の事項について、基礎的な事項をきちんと押さえながら、直感的に理解していくことを目指して説明を行います。

以下のような点に気を付けて、進めていきたいと考えています。
\begin{boxnote}
	\large
	\begin{itemize}
		\item 進め方のポイント
		\begin{itemize}	
			\item
			  イメージしやすい、直感的な理解を目指す。
			  \begin{itemize}
			  \item
				全体を俯瞰した概念的な説明
			  \item
				多様な切り口からの説明
			  \end{itemize}
			\item
			  大事なことは何度か繰り返す。
			  \begin{itemize}
			  \item
				一度ではわかりにくいかも。
			  \item
				似たような内容を、ちょっと違う言葉で。
			  \end{itemize}
			\item
			  ゆっくり議論
		\end{itemize}
		\item 数式に頼らないという意味
		\begin{itemize}
			\item 状態をイメージするためには、数学的な感覚は有効
				\begin{itemize}
					\item 直感的に感じるイメージを数学とつなげて理解すれば、\\
					共有しやすい。
					\item そのために、数学(算数?)的な事項の復習もやります。
				\end{itemize}
			\item 「天下りの数式展開」は無駄。
				\begin{itemize}
				\item 意味の理解できない数式は無意味。
				\item たいてい、思考停止を招くだけ。
				\item 数式の表す内容をイメージしましょう。
				\end{itemize}
		\end{itemize}
	\end{itemize}
\end{boxnote}

また、本講座では、できるだけ数式には頼らない説明を目指します。

ただ、「数式に頼らない」ということは、「数式を使わない」という意味ではありません。
本論でお話しますように、我々がレオロジーを使いこなして行くためには、対象となる物質の状態をきちんとイメージすることが重要になります。
そのイメージングの際に、数学的な感覚は非常に役に立つものなのです。

私に分かり易い表現があなたにとっても有効とは限らないので、数式の表すものを一緒に考えていきましょう。