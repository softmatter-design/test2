\documentclass[uplatex,dvipdfmx,a4paper,11pt]{jsarticle}

\usepackage{docmute}


% 数式
\usepackage{amsmath,amsthm,amssymb}
\usepackage{bm}
% 画像
\usepackage{graphicx}

\usepackage{multirow}
\usepackage{wrapfig}
\usepackage{ascmac}
\usepackage{xcolor}


\usepackage{makeidx}
\makeindex

\graphicspath{{../../_Figures//}{../../_Figures/Rheology/}}

\usepackage{qrcode}
\setlength\lineskiplimit{0pt}
\setlength\normallineskiplimit{0pt}

\usepackage{qexam}

\usepackage{titlesec}
\titleformat*{\section}{\Large\bfseries}
\titleformat*{\subsection}{\large\bfseries}
\titleformat*{\subsubsection}{\normalsize\bfseries}
\titleformat*{\paragraph}{\normalsize\bfseries}

% ページ設定
% \pagestyle{empty}
% 高さの設定
\setlength{\textheight}{\paperheight}   % ひとまず紙面を本文領域に
\setlength{\topmargin}{-5.4truemm}      % 上の余白を20mm(=1inch-5.4mm)に
\addtolength{\topmargin}{-\headheight}  % 
\addtolength{\topmargin}{-\headsep}     % ヘッダの分だけ本文領域を移動させる
\addtolength{\textheight}{-40truemm}    % 下の余白も20mmに%% 幅の設定
\setlength{\textwidth}{\paperwidth}     % ひとまず紙面を本文領域に
\setlength{\oddsidemargin}{-5.4truemm}  % 左の余白を20mm(=1inch-5.4mm)に
\setlength{\evensidemargin}{-5.4truemm} % 
\addtolength{\textwidth}{-40truemm}     % 右の余白も20mmに
% 図と本文との間
%\abovecaptionskip=-5pt
%\belowcaptionskip=-5pt
%
% 全体の行間調整
% \renewcommand{\baselinestretch}{1.0} 
% 図と表
%\renewcommand{\figurename}{Fig.}
%\renewcommand{\tablename}{Tab.}
%

% \makeatletter 
% \def\section{\@startsection {section}{1}{\z@}{1.5 ex plus 2ex minus -.2ex}{0.5 ex plus .2ex}{\large\bf}}
% \def\subsection{\@startsection{subsection}{2}{\z@}{0.2\Cvs \@plus.5\Cdp \@minus.2\Cdp}{0.1\Cvs \@plus.3\Cdp}{\reset@font\normalsize\bfseries}}
% \makeatother 

\usepackage[dvipdfmx,%
 bookmarks=true,%
 bookmarksnumbered=true,%
 colorlinks=false,%
 setpagesize=false,%
 pdftitle={数式に頼らない直感的理解による材料設計のためのレオロジー⼊⾨},%
 pdfauthor={佐々木裕},%
 pdfsubject={},%
 pdfkeywords={レオロジー; 材料設計; }]{hyperref}
\usepackage{pxjahyper}

\usepackage{plext}

\usepackage{niceframe} 
\usepackage{framed}
\newenvironment{longartdeco}{%
  \def\FrameCommand{\fboxsep=\FrameSep \artdecoframe}%
  \MakeFramed {\FrameRestore}}%
 {\endMakeFramed}
 
\usepackage{siunitx}

\newcommand{\rmd}{\mathrm{d}}

\usepackage[inline]{showlabels}

\begin{document}

\question{演習問題 1}
内容を振り返るために、以下に示した文章例の中から適切な記述のものを複数選んでください。
\begin{qlist}
	\qitem 「関数」と「グラフ」というそれぞれの考え方を表す正しい言葉はどれでしょうか?
		\begin{qlist2}
			\qitem 関数とは、入力と出力との間の関係を表している変換装置のようなものと捉えられます。
			\qitem 関数とは、数の集合に値を取る写像の一種です。
			\qitem 関数とは、ものの関係を表す数のことを指します。
			\qitem グラフとは、入力と出力との関係を視覚的に理解しやすくしたものです。
			\qitem グラフを用いることで、関数の出力の値を正確に読み取ることができます。
		\end{qlist2}
    \vspace{3mm}
	\qitem 「線型性」と「物理モデル」というそれぞれの考え方を表す正しい言葉はどれでしょうか?
		\begin{qlist2}
			\qitem 線型性とは、グラフに表した時に原点を通る直線となるような性質であり、比例の関係を表します。
			\qitem 線型性とは、放物線と呼ばれる曲線で表される性質であり、反比例の関係を表します。
			\qitem 物理モデルとは、「事象や理論の成り立ちを説明するための簡単で理解しやすい概念や模型」です。
			\qitem 我々の身の回りに起こっている実際の事柄は、非常に単純で線型で記述できる場合がほとんどです。
			\qitem 入力が小さい場合には応答が線型で取り扱える場合が多いことが知られています。
    \end{qlist2}
    \vspace{3mm}
	\qitem 「量」について正しい記述を選んでください。
		\begin{qlist2}
			\qitem 量とは、定性的に区別でき、かつ、定量的に決定できるものです。
			\qitem 同じ種類の量同士は「和と差」の演算が定義できて、結果は同じ種類の量となります。
			\qitem 異なる種類の量であっても、いかなる演算でもできます。
			\qitem 同じ、あるいは、異なる種類の量同士でも積や商が定義できる場合もあります。
			\qitem 長さ同士の積は、体積を表します。
    \end{qlist2}
    \vspace{3mm}
	\qitem 「次元」について正しい記述を選んでください。
		\begin{qlist2}
			\qitem 次元とは、注目する「ある量」が、どのような現象であるかを「基本量の積と商で表す」ような考え方といえます。
			\qitem 面積という量は、長さという基本量が掛け合わされることで、広さという現象を表しています。
			\qitem 次元とは、物質の性質を表す量のことです。
			\qitem 次元の関係式とは「定数係数を無視した等式として表すことで物理現象の成り立ちを表して」います。
			\qitem ここで用いた次元という考え方は、一般に使われている三次元や四次元という言葉とは全く関係ありません。
		\end{qlist2}
    \vspace{3mm}
	\qitem 「単位」について正しい記述を選んでください。
		\begin{qlist2}
			\qitem 単位とは、「量の大きさを表すため」に特定の会社間で取り決めによって定義されたものです。
			\qitem 単位とは、「同種の物理量の大きさを表すため」に取り決めによって定義されたものです。
			\qitem 現在、最も広く使われている単位系は、国際単位系(SI)です。
			\qitem JIS と呼ばれる単位系は、日本で広く使われています。
			\qitem 任意の物理量の値 $Q$ は、その大きさを表す数値 $n$ と単位 $U$ との積として表されることになります。
		\end{qlist2}
\end{qlist}

\question{演習問題 2}
内容を振り返るために、テキストで用いた言葉を使って簡単な穴埋めを行ってください。
\begin{qlist}
  \qitem 量の次元に関して、国際量体系で表のように7つの基本量が定められています。レオロジーでよく使う四つの基本量を、
  以下の\qbox{(a)}から\qbox{(d)}までのカッコを埋めてください。
	\begin{center}
		\begin{tabular}{|c|c||c|c|} \hline
			基本量 		& 次元の記号 & SI単位 		& 単位の記号\\ \hline \hline
			\qbox{}		& L			& メートル 		& m \\ \hline
			\qbox{}		& M			& キログラム 	& kg \\ \hline
			\qbox{}		& T			& 秒 			& s \\ \hline
			電流		& I			& アンペア 		& A \\ \hline
			\qbox{}	& $\Theta$	& ケルビン 		& K \\ \hline
			物質量		& N			& モル 			& mol \\ \hline
			光度		& J			& カンデラ 		& cd \\ \hline
		\end{tabular}
	\end{center}

	\qitem 以下に示した組立単位について、以下の\qbox{(e)}から\qbox{(i)}までのカッコを埋めてください。
	\begin{center}
		\begin{tabular}{|c|c||c|c|} \hline
			組立量 		& 名称					& 記号		& SI 基本単位による表現 	\\ \hline \hline
			\qbox{}		& ヘルツ (hertz)		& Hz		&  s$^{-1}$ 					\\ \hline
			\qbox{}		& ニュートン (newton)	& N 		& m$\cdot$kg$\cdot$s$^{-2}$ 	\\ \hline
			\qbox{}		& パスカル (pascal)		& Pa 		& (N/m$^2$) = m$^{-1}\cdot$kg$\cdot$s$^{-2}$ \\ \hline
			\qbox{}	& ジュール (joule)		& J 		& (N$\cdot$m) = m$^{2}\cdot$kg$\cdot$s$^{-2}$ \\ \hline
			\qbox{}		& パスカル秒			& Pa$\cdot$s & m$^{-1}\cdot$kg$\cdot$s$^{-1}$ \\ \hline
		\end{tabular}
  \end{center}
  
  \begin{itembox}[l]{選択肢}
    \begin{center}
      \begin{tabular}{lllll}
        1. 応力&2. 質量&3. 時間&4. エネルギー&5. 粘度\\
        6. 周波数&7. 長さ&8. 熱力学温度&9. 力
      \end{tabular}
    \end{center}
  \end{itembox}
\end{qlist}

\question{演習問題 3}
数行程度の簡単な記述で構いませんので、以下の自由記述問題を考えてみてください。
\begin{qlist}
\qitem 「レオロジー」という考え方を理解して、多様な場面において使いこなすためには、
「モデル化」という考え方がとても大事だと筆者は強く感じています。\\
皆さんがモデル化ということに対して感じることを書いてみてください。
\end{qlist}

\end{document}