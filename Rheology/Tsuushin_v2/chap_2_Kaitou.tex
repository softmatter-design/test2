\begin{document}

\question{解答欄 1}
\begin{table}[htb]
    \begin{center} 
      \begin{tabular}{|p{.15\textwidth}|p{.15\textwidth}|p{.15\textwidth}|p{.15\textwidth}|p{.15\textwidth}|} \hline
        (1) & (2) & (3) & (4) & (5)\\ \hline \hline
          &  & & &  \\ \hline		
      \end{tabular}
    \end{center}
  \end{table}

  \question{解答欄 2}
  \begin{table}[htb]
    \begin{center} 
      \begin{tabular}{|p{.08\textwidth}|p{.08\textwidth}|p{.08\textwidth}|p{.08\textwidth}|p{.08\textwidth}|p{.08\textwidth}|p{.08\textwidth}|p{.08\textwidth}|p{.08\textwidth}|} \hline
        (a) & (b) & (c) & (d) & (e) & (f) & (g) & (h) & (i)\\ \hline
          &  & & & & & & &  \\ \hline		
      \end{tabular}
    \end{center}
  \end{table}

\question{解答欄 3}
\begin{table}[htb]
  % \renewcommand\arraystretch{2.0}
  \begin{center} 
    \begin{tabular}{|l|p{.8\textwidth}|} \hline
      & \\
      私にとっての  & \\
      モデル化とは  & \\ 
      & \\ \hline
    \end{tabular}
  \end{center}
\end{table}

\end{document}